\documentclass[]{book}
\usepackage{vub}
\usepackage{amsfonts}
\usepackage{amsmath}
\usepackage{pgfplots}
\usepackage{bm}
\usepackage{amssymb}
\usepackage{graphicx}
\usepackage{tikz}
\usepackage{subcaption}
\usepackage[utf8]{inputenc}
\usepackage{geometry}
\usepackage{multirow}
\usepackage{array, makecell, rotating}
\usepackage{cleveref}
\usepackage{natbib}
\usepackage[toc,page]{appendix}
\usepackage{listofitems} % for \readlist to create arrays
\usetikzlibrary{arrows.meta} % for arrow size
\usepackage[outline]{contour} % glow around text
\contourlength{1.4pt}
\tikzstyle{mynode}=[thick,draw=blue,fill=blue!20,circle,minimum size=22]


%TODO UITLEG DISENTANGLEMENT:
%"Disentangled representation is a method that disentangles or represents each feature into narrowly defined variables and encodes them as separate dimensions [1]." cite latifDeepRepresentationLearning2021
%Y. Bengio, A. Courville, and P. Vincent, “Representation learning: A review and new perspectives,” IEEE transactions on pattern analysis and machine intelligence, vol. 35, no. 8, pp. 1798–1828, 2013


\tikzset{>=latex} % for LaTeX arrow head
\usepackage{xcolor}
\colorlet{myred}{red!80!black}
\colorlet{myblue}{blue!80!black}
\colorlet{mygreen}{green!60!black}
\colorlet{myorange}{orange!70!red!60!black}
\colorlet{mydarkred}{red!30!black}
\colorlet{mydarkblue}{blue!40!black}
\colorlet{mydarkgreen}{green!30!black}
\tikzstyle{node}=[thick,circle,draw=myblue,minimum size=22,inner sep=0.5,outer sep=0.6]
\tikzstyle{node in}=[node,green!20!black,draw=mygreen!30!black,fill=mygreen!25]
\tikzstyle{node hidden}=[node,blue!20!black,draw=myblue!30!black,fill=myblue!20]
\tikzstyle{node convol}=[node,orange!20!black,draw=myorange!30!black,fill=myorange!20]
\tikzstyle{node out}=[node,red!20!black,draw=myred!30!black,fill=myred!20]
\tikzstyle{connect}=[thick,mydarkblue] %,line cap=round
\tikzstyle{connect arrow}=[-{Latex[length=4,width=3.5]},thick,mydarkblue,shorten <=0.5,shorten >=1]
\tikzset{ % node styles, numbered for easy mapping with \nstyle
	node 1/.style={node in},
	node 2/.style={node hidden},
	node 3/.style={node out},
}
\def\nstyle{int(\lay<\Nnodlen?min(2,\lay):3)} % map layer number onto 1, 2, or 3


\usetikzlibrary{shapes.geometric, arrows}
\usepackage[pdftex]{pict2e}




%\includeonly{background}
%\includeonly{variationalcontrastivepredictivecoding}
%\includeonly{experiments}
%\includeonly{relatedwork}
%\includeonly{appendix}





\pagenumbering{gobble}




%\title{Variational Greedy InfoMax: Interpretable Representation Learning with Latent Space Constraints}
%\subtitle{Combining Contrastive Predictive Coding and Variation Autoencoders for Enhanced Self-Supervised Learning}

\title{Variational Greedy InfoMax}
\subtitle{Interpretable representation learning with latent space constraints}

\author{Fabian Denoodt}
\faculty{Science and Bio-Engineering Sciences}
\promotors{Promotor(s):~Prof. Dr. Bart de Boer}
\pretitle{Master thesis submitted in partial fulfilment of the requirements for the degree of Master of Science In de Ingenieurswetenschappen: Computerwetenschappen}
\date{2022-2023}


\begin{document}
\maketitle
\title{[Dutch] Variational Greedy InfoMax}
\subtitle{[Dutch] Towards independent and interpretable representations}
\pretitle{Proefschrift ingediend met het oog op het behalen van de graad van Master of Science In de Ingenieurswetenschappen: Computerwetenschappen}
\faculty{Wetenschappen en Bio-ingenieurswetenschappen}
\maketitle

\tableofcontents


%%%%%%%%%%%%%%%%%%%%%%%%%%%%%%%%%%%%%%%%%%%%%%%%%%%%%%%%%%%%%%%%%%%%%%
% LaTeX Overlay Generator - Annotated Figures v0.0.1
% Created with http://ff.cx/latex-overlay-generator/
% If this generator saves you time, consider donating 5,- EUR! :-)
%%%%%%%%%%%%%%%%%%%%%%%%%%%%%%%%%%%%%%%%%%%%%%%%%%%%%%%%%%%%%%%%%%%%%%
%\annotatedFigureBoxCustom{bottom-left}{top-right}{label}{label-position}{box-color}{label-color}{border-color}{text-color}
\newcommand*\annotatedFigureBoxCustom[8]{\draw[#5,thick,rounded corners] (#1) rectangle (#2);\node at (#4) [fill=#6,thick,shape=circle,draw=#7,inner sep=2pt,font=\sffamily,text=#8] {\textbf{#3}};}

%\annotatedFigureBox{bottom-left}{top-right}{label}{label-position}
\newcommand*\annotatedFigureBox[4]{\annotatedFigureBoxCustom{#1}{#2}{#3}{#4}{white}{white}{black}{black}}

%\newcommand*\annotatedFigureText[4]{\node[draw=none, anchor=south west, text=#2, inner sep=0, text width=#3\linewidth,font=\sffamily] at (#1){#4};}

\newcommand*\annotatedFigureText[5]{\node[draw=none, anchor=south west, text=#2, inner sep=0, text width=#3\linewidth,font=\sffamily\fontsize{#5}{14}\selectfont] at (#1){#4};}


\newenvironment {annotatedFigure}[1]{\centering\begin{tikzpicture}
		\node[anchor=south west,inner sep=0] (image) at (0,0) { #1};\begin{scope}[x={(image.south east)},y={(image.north west)}]}{\end{scope}\end{tikzpicture}}
%%%%%%%%%%%%%%%%%%%%%%%%%%%%%%%%%%%%%%%%%%%%%%%%%%%%%%%%%%%%%%%%%%%%%%


\newcommand{\vect}[1]{\mathbf{#1}}
\newcommand{\vecti}[1]{\mathbf{#1}^{(i)}}
\newcommand{\vectj}[1]{\mathbf{#1}^{(j)}}

\newcommand{\xith}[0]{\vecti{x}}
\newcommand{\zith}[0]{\vecti{z}}

\newcommand{\kl}[2]{D_{KL} \left( #1 \mid \mid #2 \right)}
\newcommand{\condp}[2]{p(#1 \mid #2)} % conditional probability
\newcommand{\condq}[2]{q(#1 \mid #2)} % conditional probability
\newcommand{\probzx}[0]{\condp{\vect{z}}{\vecti{x}}} % conditional probability
\newcommand{\probxz}[0]{\condp{\vecti{x}}{\vect{z}}} % conditional probability
\newcommand{\qprobzx}[0]{\condq{\vect{z}}{\vecti{x}}} % conditional probability
\newcommand{\qprobqxz}[0]{\condq{\vecti{x}}{\vect{z}}} % conditional probability

%_{x \in p_{data}(x),\\ z \in p_{z}(z)}}
\newcommand{\expected}[1]{\mathop{\mathbb{E}}_{\substack{#1}}}

\newcommand{\naturalset}[0]{\mathbb{N}}

\newcommand{\sample}[2]{#1 \sim #2}
\newcommand{\expectedsample}[2]{\expected{\sample{#1}{#2}}}
\newcommand{\expectedsamplezq}[0]{\expectedsample{\vecti{z}}{q (\cdot \mid \vecti{x})}}
%\newcommand{\expectedsamplezq}[0]{\expectedsample{\vect{z}}{q_\phi (\cdot \mid \vecti{x})}}

\newcommand{\pzx}[0]{\probzx}
\newcommand{\pxz}[0]{\probxz}

\newcommand{\qzx}[0]{\qprobzx}
\newcommand{\qxz}[0]{\qprobxz}




% p_theta(z|x)





\newcommand{\qz}[0]{p(\vect{z})}
\newcommand{\pz}[0]{p_\theta(\vect{Z})}
%\newcommand{\pzblank}[0]{p_\theta(\cdot)}
\newcommand{\pzblank}[0]{p(\cdot)}


\newcommand{\px}[0]{p( \vecti{x} )}

\newcommand{\elbo}[0]{  \expectedsamplezq \left[ \log p(\vect{Z}, \vecti{x}) - \log \qzx \right]  }
\newcommand{\logevidence}[0]{\log \px}

\newcommand{\lelbo}[0]{\mathcal{L}_{\textit{ELBO}}}

\newcommand{\klapproxposterior}[0]{\kl{\qprobzx}{\probzx}}


\newcommand{\reconstr}[0]{\expectedsamplezq \left[ -\log \pthetaxzi \right]}



\newcommand{\latentspaceconstraint}[0]{\kl{ \qphizxblank }{ \pzblank } }

\newcommand{\latentspaceconstraintstandardgaussian}[0]{\kl{ \qphizxblank }{ \standardnormal } }

\newcommand{\latentspaceconstraintclosedform}[0]{ \frac{1}{2} \sum_{k=1}^{D} \left( -\log (\sigma_k^{(i)})^2 - 1 + (\sigma_k^{(i)})^2 +  (\mu_k^{(i)})^2 \right) }


\newcommand{\latentspaceconstraintclosedformNoI}[0]{ \frac{1}{2} \sum_{k=1}^{D} \left( -\log \sigma_k^2 - 1 + \sigma_k^2 +  \mu_k^2 \right) }


\newcommand{\elboexplicit}[0]{ \reconstr + \latentspaceconstraint }
\newcommand{\betaelboexplicit}[0]{ \reconstr + \beta~ \latentspaceconstraint }

\newcommand{\mufat}[0]{\bm{\mu}}
\newcommand{\sigmafat}[0]{\bm{\sigma}}

\newcommand{\mui}[0]{\bm{\mu}^{(i)}}
\newcommand{\sigmai}[0]{\bm{\sigma}^{(i)}}
\newcommand{\epiloni}[0]{\bm{\epsilon}^{(i)}}

\newcommand{\epilonfat}[0]{\bm{\epsilon}}

\newcommand{\identitymtx}[0]{\mathbf{I}}
\newcommand{\covariancemtx}[0]{\mathbf{\Sigma}}

%\newcommand{\sigmaisq}[0]{(\bm{\sigma}^{(i)})^2 }
\newcommand{\diagsigmai}[0]{\text{diag}(\sigmai)} % maybe should be sigmaisq
\newcommand{\diagsigma}[0]{\text{diag}(\sigmafat)} % maybe should be sigmaisq
\newcommand{\normal}[0]{\mathcal{N}(\mui, \diagsigmai)}
\newcommand{\normalNoI}[0]{\mathcal{N}(\mufat , \diagsigma)}
\newcommand{\standardnormal}[0]{\mathcal{N}(\bm{\vect{0}}, \identitymtx)}

\newcommand{\qphizx}[0]{q (\vect{z} \mid \vecti{x})}
\newcommand{\pphizx}[0]{p (\vect{z} \mid \vecti{x})}
%\newcommand{\qphizx}[0]{q_\phi (\vect{z} \mid \vecti{x})}

\newcommand{\qvae}[0]{q (\vect{z} \mid \vect{x})}
\newcommand{\qvaei}[0]{q (\vecti{z} \mid \vecti{x})}
\newcommand{\qvaeEmptyZandXI}[0]{q ( \cdot \mid \vecti{x})}

\newcommand{\qphizxblank}[0]{q (\cdot \mid \vecti{x})}
%\newcommand{\qphizxblank}[0]{q_\phi (\cdot \mid \vecti{x})}

%\newcommand{\pthetaxz}[0]{p_\theta (\vecti{x} \mid \vect{z})}
\newcommand{\pthetaxz}[0]{p (\vecti{x} \mid \vect{z})} % TODO: I REMOVED THETA BECAUSE I DON'T UNDERSTAND IT.
\newcommand{\pthetaxzi}[0]{p (\vecti{x} \mid \vecti{z})}


\newcommand{\tildexi}[0]{\widetilde{\mathbf{x}} ^ {(i)}}
\newcommand{\tildex}[0]{\widetilde{\mathbf{x}}}

\newcommand{\samplestandardnormal}[1]{\sample{#1}{\standardnormal}}


\newcommand{\W}{\mathbf{W}}	

%cpc

\newcommand{\fzc}[0]{f_k(\ztk, \ct)}
\newcommand{\pztk}[0]{p(\ztk)}
\newcommand{\pzcdivpz}[0]{\frac{p(\ztk \mid \ct)}{\pztk}}

\newcommand{\mutinfztkct}[0]{I(z_{t+k}, c_t)}


\newcommand{\x}[0]{\vect{x}}
\newcommand{\xt}[0]{\x_t}
\newcommand{\xtk}[0]{\vect{x}_{t+k}}
\newcommand{\xtone}[0]{\vect{x}_{t+1}}

\newcommand{\z}[0]{\vect{z}}
\newcommand{\zt}[0]{\z_t}
\newcommand{\zi}[0]{\z_i}
\newcommand{\zj}[0]{\z_j}
\newcommand{\ztk}[0]{\vect{z}_{t+k}}
\newcommand{\ztone}[0]{\vect{z}_{t+1}}

\newcommand{\ct}[0]{\vect{c}_t}
\newcommand{\ctk}[0]{\vect{c}_{t+k}}
\newcommand{\ctone}[0]{\vect{c}_{t+1}}

\newcommand{\fkdefinition}[0]{f_k(\z_j, \ct) = \exp(\z_j^T W_k \ct)}

\newcommand{\fkzc}[0]{f_k(\z_{t+k}, \ct)}

\newcommand{\nceprediction}[0]{\frac{\fkzc}{\sum_{\zj \in X} f_k(\zj, \ct)}}



% GIM

\newcommand{\fkblank}[0]{f_k(\cdot)}
\newcommand{\fkmblank}[0]{f_k^m(\cdot)}

\newcommand{\fkm}[0]{f_k^m(\ztk^m,\zt^m)}


\newcommand{\gim}[0]{\log \frac{\fkm }{\sum_{\zj^m \in X} f_k^m(\zj^m, \zt^m)}}


\newcommand{\R}[0]{\mathbb{R}}

\newcommand{\qzzblank}[0]{q(\cdot \mid \zt^{m-1})}

\newcommand{\latentspaceconstraintgim}[0]{\kl{ \qzzblank }{ \standardnormal } }

\newcommand{\sampleqdot}[1]{q(\cdot \mid #1)}


\newcommand{\reconstrgim}[0]{\sum_k
\expected{
	\sample{\ztk^m}{\sampleqdot{\ztk^{m-1}} } \\ 
	\sample{\zt^m}{\sampleqdot{\zt^{m-1}}} } 
\left[ \gim \right]
}


\newcommand{\genc}[0]{g_{enc}(\cdot)}
\newcommand{\gencm}[0]{g_{enc}^m(\cdot)}
\newcommand{\gencM}[0]{g_{enc}^M(\cdot)}

\newcommand{\gar}[0]{g_{ar}(\cdot)}

\newcommand{\Lvnce}[0]{\mathcal{L}_{\text{V-NCE}}}
\newcommand{\Lnce}[0]{\mathcal{L}_{\text{NCE}}}

\newcommand{\qfromzmneg}[0]{\sampleqdot{\zt^{m-1}}}
\newcommand{\qfromzm}[0]{\sampleqdot{\zt^m}}
\newcommand{\qfromzM}[0]{\sampleqdot{\zt^M}}

\newcommand{\normalfatmusigma}[0]{\mathcal{N}(\mufat, \text{diag}(\sigmafat^2))}



\newcommand{\D}[0]{\mathcal{D}}
\newcommand{\X}[0]{\mathcal{X}}
\newcommand{\Z}[0]{\mathcal{Z}}
\newcommand{\Y}[0]{\mathcal{Y}}
\newcommand{\f}[0]{f(\cdot) }
\newcommand{\T}[0]{T(\cdot) }
\newcommand{\E}[0]{E(\cdot) }

\newcommand{\Dtrain}[0]{\D_{\text{train}}}

\newcommand{\xtith}[0]{\xt^{(i)}}
\newcommand{\xtjth}[0]{\xt^{(j)}}
\newcommand{\ztm}[0]{\zt^{m}}
\newcommand{\ztmnegone}[0]{\zt^{m-1}}

\newcommand{\MEL}[0]{\text{MEL}}

\newcommand{\prior}[0]{p(\z)}
\newcommand{\priorblank}[0]{p( \cdot )}

\chapter{Introduction}
%https://www.scribbr.com/dissertation/introduction-structure/

\setcounter{page}{1} % Set the page counter to 3
\pagenumbering{arabic}


In recent years, CPC has been shown to be a successful self-supervised learning approach in a wide range of domains \cite{stackeEvaluationContrastivePredictive2020, dehaanContrastivePredictiveCoding2021, luSemiSupervisedHistologyClassification2019, bhatiSegmentalContrastivePredictive2021b, deldariTimeSeriesChange2021, henaffDataEfficientImageRecognition2020}. Löwe et al. contribute to this work by showing that modules can be trained greedily, each with their own instance of the InfoNCE loss, enabling modules to be trained in parallel, or sequentially. Moreover, Wang \cite{meihanwangSpeechRepresentationLearning2019} empirically demonstrates that GIM learns good speech representations for the speech dataset we use in our own experiments \ref{cha:experim_details_vgim}.

However, the underlying representations obtained from optimising this InfoNCE objective have not yet been studied in great detail. In this thesis we aim to gain better understanding in these representations by enforcing constraints to the latent space, resulting in a space that is easier to analyse and understand.






%What is a literature review?
%• A literature review generally also presents a short historical overview
%	• How did the field become the way it is?
%	• What is the effect of history?

%• However, the important thing is that it shows why and how your research is important
%	• So it needs to identify open questions and hot topics




%Referring with author and year
%• (Young 2010)
%• As shown by Young (2010) ...
%• (Stills & Young 1976)
%• (Crosby, Stills, Nash & Young 1969)
%• (Crosby et al. 1969) – subsequently
%• or always et al. if more than six authors
%• (Young 2010, 2012)
%• (Young 1975a, 1975b)
%• (Stills & Young 1976; Young 2010)













--------------


% TODO:
FEEDBACK BART

!!!
wat is + waarom het gedaan wordt.

should also explain in my work.

should be around 60 pages.

Discussion:
Sectie 4.5: (Bart wil aparte sectie)
- Hoe helpt deze techniek om het netwerk interpreteerbaarder te maken.
- compare techniques: whether they are better explainable.
- Zou na results moeten zijn. en kan zo inleiding zijn naar future work.
- bart zou verwachten dat cha 3 zeer groot is.

- UVA: wordt geschreven voor begeleiders, hun hebben meer achtergrond.
- Jury aan VUB andere verwachtingen, wil meer uitleg.

- mijn discussie sectie moet langer! en zeker in vertellen waarom het beter is voor visualities.

---
Defense:
- verduideluikende vragen
- critiek als gaten in argumentatie
- hun komen met suggestie die ze uit literatuur kennen, en moet bv mening over geven.
- "zou dat ook anders gekund hebben"
!!
---

%Order of importance: 
%Solution + evaluation, introduction + related work, 
%Fundamentals Future work


%\begin{itemize}
%	\item Small section on biological neural networks leading into Sindy's gradient isolated learning
%	\item Neural networks lack explainability, important for use cases. E.g., our eventual use case: to see how a neural network learns speech, and if we can understand that, it may lead to insights in how humans do speech.
%	\item Our contributions
%\end{itemize}

% Overview of thesis
\begin{itemize}
	\item Context: !E GIM for representation learning: generates representations that simplify classification tasks vs when done on raw data
	\item Problem: If wants to know for what tasks applicable... must know what is present in data
	\item Solution: Analysis of learned representations on speech data
	\item My contributions: 
	\begin{itemize}
		\item Decoder ANN for each layer of GIM: Shows what information is maintained through the layers
		\item Search correlations between kernels weights and signal features
		\item Extension on CPC via VAE
	\end{itemize}
\end{itemize}



% what a background is [generic]
%\begin{itemize}
%	\item Explains "what is" questions
%	\item State of the art neural networks take raw data as input and do pre-processing automatically
%	\item Explanation of Sindy's neural network
%	\item Techniques behind what I do
%\end{itemize}
%\hrule

\chapter{Background}
\section{Entropy, relative entropy and mutual information}
We first discuss specific definitions from information theory. These concepts will be relevant to understand contrastive predictive coding, which we discuss in a following section. The formal definitions are obtained from the book "Elements of Information Theory" \cite{coverELEMENTSINFORMATIONTHEORY}. The equations that contain a log function are assumed to be under base two.
\subsection{Shannon's Entropy}
Entropy measures the average amount of information required to describe a random variable \cite{coverELEMENTSINFORMATIONTHEORY}. The entropy H(X) of a discrete random variable $X$, is formally defined in equation \ref{eq:entropy} shown below. 

\begin{equation}
	H(X) = -\sum_{x\in\mathcal{X}} p(x) \log p(x)  \label{eq:entropy}
\end{equation}

The alphabet $\mathcal{X}$ represents the set of events the random variable X can take, formerly known as the \textit{sample space}. Additionally, $p: \mathcal{X} \rightarrow [0, 1]$ denotes the probability density function of $X$. Hence, given an event $ x \in \mathcal{X}, p(x) $ corresponds to the probability of event x occurring.

%[general explanation]
Assume a random variable $X$ with possible events $x_1$, $x_2$. Intuitively, when $p(x_1)$ is low, the surprise when the event $x_1$ occurs will be high. The surprise for one event is denoted in equation \ref{eq:surprise}. 

\begin{equation}
	-p(x) \log p(x) \label{eq:surprise}
\end{equation}

Hence, entropy can also be considered as the sum of “surprise” over each event \cite{datasciencecoursesAliGhodsiLec2017}. 
%[algebraic explanation]
To understand why equation \ref{eq:surprise} does indeed correspond to a measure of surprise, consider an event $x \in \mathcal{X}$ with $p(x) = 1$. Note that $\log p(x) = 0$, and thus the surprise is zero. Meanwhile, if p(x) approaches $0$, $\log p(x)$ goes to $- \inf$. And hence, by the negation sign in formula \ref{eq:entropy} the surprise is large.


\begin{figure}[h]
	\centering
	\includegraphics[width=0.4\linewidth]{screenshot005}
	\caption{$H(p)$ vs $p$ (originates from "Elements of Information Theory", page 16)}
	\label{fig:EntropyvsP}
\end{figure}



%[book: p16: note should rename figure to axes to ” H(x) vs …”]
Figure \ref{fig:EntropyvsP} displays when entropy reaches its maximum for the case of a random variable with 2 outcomes. We can see that the entropy, and thus, the information is largest when the probability of the two outcomes is equal to each other, namely $p(x_1)=p(x_2)=0.5$. Note that for a random variable $X$ with more than two events, $H(X)$ can be larger than one.

\subsection{Relative entropy and mutual information}
Relative entropy, also known as the Kullback Leibler (KL) divergence, is defined in equation \ref{eq:kl}, where $p$ and $q$ denote a probability density function over the same sample space $\mathcal{X}$ \cite{coverELEMENTSINFORMATIONTHEORY}. The KL divergence quantifies the “divergence” or "distance" between the two distributions. Note that $D(p||q)$ does not necessarily correspond to $D(q||p)$ and thus the metric is not symmetrical.

\begin{equation}
	\kl{q}{p} = \sum_{x\in\mathcal{X}}p(x) \log \frac{p(x)}{q(x)} \label{eq:kl}
\end{equation}



% was rephrased by chat gpt
The mutual information (MI) between two random variables $X$ and $Y$ can be computed as the KL divergence between their joint probability distribution, $p_{X,Y}(x,y)$, and the product of their marginal probability distributions, $p_X(x)$ and $p_Y(y)$, which is denoted as $p_X(x)p_Y(y)$ \cite{coverELEMENTSINFORMATIONTHEORY}. The equation for mutual information then becomes:

\begin{equation}
	I(X; Y) =  \kl{p_{X, Y}(x, y)}{p_X(x) p_Y(y)}
\end{equation}

As described by Cover and Thomas in their book "Elements of Information Theory" \cite{coverELEMENTSINFORMATIONTHEORY}, I(X; Y) quantifies the amount of information $Y$ describes about $X$. A alternative definition for $I(X;Y)$ is illustrated in \ref{eq:MI_reduce}. The equation provides us with an intuitive meaning for $I(X;Y)$, corresponding to the surprise caused by $X$, which is reduced by the knowledge of $Y$. In a following section, we discuss how these concepts from information theory are applied in representation learning, by maximising the mutual information between latent representations.

\begin{equation}
	I(X;Y)= H(X) - H(X|Y) \label{eq:MI_reduce}
\end{equation}



\section{Supervised neural networks}

% COMMENT CHAT GPT: WHAT IS THE PURPOSE OF THIS SECTION?
%Clarify the purpose of the section: The section starts by stating that it will discuss traditional supervised learning approaches, but it's not entirely clear what the purpose of the section is. Are you providing an introduction to supervised learning for readers who may be unfamiliar with it? Or are you focusing on the specifics of how supervised learning is implemented with ANNs?

We shall now discuss traditional supervised learning approaches, as these will lay the groundwork for the representation learning approaches discussed in the following section. 

Typical supervised machine learning problems consider the problem where given a training set of tuples ($x_i$, $y_i$), a mapping function $f: \mathcal{X} \rightarrow \mathcal{Y}$ must be learned, where $x_i \in \mathcal{X}$ corresponds to the input sample and $y_i \in \mathcal{Y}$ the label. A good $f$ will then also generalise well to unseen $x_i \in \mathcal{X}$, which were not part of the training set.

Artificial Neural Networks (ANNs) in particular, tackle this problem by defining $f$ as a fixed set of parameters consisting of a series of layers. During inference, at each layer $l$ a transformation matrix $W^l$ is applied to the output vector from the previous layer $a^{l-1}$. This is shown in the equation below.

$$ z^l = W^l a^{l-1} $$

Secondly, a non-linear function $\sigma: R^d \rightarrow R^d$ is applied to $z^{l}$, as shown in equation \ref{eq:nonlinearity}. The resulting vector $a^l$ may then again be the input for a following layer.

\begin{equation}
	a^l = \sigma(z^{l})	\label{eq:nonlinearity}
\end{equation}

Hence, during inference, the input vector $x$ is propagated through each layer, resulting in a final output $\hat{y}^L$. The equation for the forward pass of a neural network with $L$ layers is described in equation \ref{eq:forward_pass}. 

\begin{equation}
	f_{W^1 ... W^L}(x) = \hat{y}^L = \sigma(… \sigma( \sigma( x^T W^1 )^T) W^2 …)^T W^L \label{eq:forward_pass}
\end{equation}

% [cost function, gradient descent, backpropagation]
During training the ANN's parameters $W^1$ ... $W^L$, are optimised according to the learning problem. The viability of the parameters is quantified by the loss function $\mathcal{L}$ over a batch of $n$ training samples, as shown in equation \ref{eq:generic_loss}. $y_i$ corresponds to the ground truth label of the single data sample $x_i$, $e(y_i, \hat{y}_i)$ the error between a ground truth $y_i$ and the estimation $\hat{y}_i$ of a single sample. The error function $e$ can be replaced depending on the task, for instance by mean squared error for regression or cross entropy for classification problems.

\begin{equation}
	\mathcal{L}(W) = \frac{1}{n}\sum_{i=1}^n e(y_i, f_{W}(x_i)) \label{eq:generic_loss}
\end{equation}

The parameters can be optimised by minimising the loss function $\mathcal{L}$ defined above. This can be done algebraically, however, it becomes difficult when the dimension of the input features is large. Gradient descent is used to find an "appropriate" minimum of the cost function, by iteratively adjusting the parameters $W^1$ ... $W^L$ via the following learning rule.

\begin{equation}
	W^l_{ij} \leftarrow W^l_{ij} - \alpha \frac{\partial \mathcal{L}}{\partial W^l_{ij}}
\end{equation}

Although gradient descent can be successfully applied for finding local minima, neural networks may still contain many parameters which each must be optimised. This is typically resolved by applying the backpropagation algorithm in combination with gradient descent, such that fewer partial derives must be calculated, resulting in more efficient optimisation.


% review from chat gpt:
%Lastly, it would be useful to provide some discussion of the limitations of supervised learning approaches, beyond the issue of requiring large amounts of labelled data. For example, it would be worth noting that these techniques can be prone to overfitting, and that they may not be well-suited to tasks where the relationship between inputs and outputs is highly complex or non-linear.


% limitations of supervised learning approaches and towards representation learning
% TODO: this paragraph should be better written
Although supervised learning though ANNs is considered successful, their performance is heavily dependent on the choice of the data representation \cite{bengioRepresentationLearningReview2013}, on which they are applied. For complex representations, a more complex architecture may be required, which requires more data to prevent overfitting. When labelled data is scarce, these models may result in seemingly well performance on the training set, but bad generalisation to data outside the training set. As a consequence, a lot of time in the machine learning pipeline is invested in transforming data into better latent representations. Good latent representations tend to disentangle the data into meaningful features, such that data can be more easily be separable into classes. In the following section we discuss how part of the labour of finding good representations can be relieved with unsupervised learning approaches, which may learn to find good latent representations. These representations could then be used as the input for supervised predictors.




\section{Representation learning through reconstruction error}
%[what is repr learning]
One of the challenges in supervised learning is the constant need of large amounts of labelled data. Hence, when a labelled dataset is small, we would like to leverage a larger unlabelled dataset as basis for learning. By doing so, a mapping can be learned from the raw input data to a representation which makes downstream tasks easier. This process of learning representations from data is commonly referred to as representation learning \cite{le-khacContrastiveRepresentationLearning2020}. Supervised learning algorithms can then learn directly from these disentangled latent representations with fewer labelled data.

In the following two sections we discuss two paradigms of representation learning with ANNs. The first paradigm is learning representations by minimising a reconstruction error. The second learns its representations by contrasting them against noise. These two paradigms will lay the basis for our own contributions in section three.

\subsection{Autoencoders}
%auto encoder: good at capturing complex, high-dimensional features of the input data
%\textbf{WHY: More interpretable latent representations through Gaussian distributions.}
Autoencoders were introduced in 1986 by D. Rumelhart et al. \cite{rumelhartLearningInternalRepresentations1988} as a means to learn compressed representations \cite{bankAutoencoders2021}. This is achieved through a neural network architecture consisting of two blocks. The first block is the encoder $E$ and receives input data which it encodes into a lower dimensional representation. The second block, called the decoder $D$, receives as input the latent representation and is tasked to reconstruct the original input. The two blocks are simultaneously optimised by \textit{minimising the reconstruction error} shown in the following equation:

\begin{equation}
	\mathcal{L} = \sum_{i = 1}^N l(\vecti{x}, D(E(\vecti{x})))
\end{equation}

where $l$ refers to the error for a single data point, for instance the $l_2$-norm and $N$ the number data points. An example autoencoder architecture is depicted in figure \ref{fig:autoencoder}. $\vect{x} \in \mathcal{D}$ corresponds to an input vector which is encoded into latent representation $\vect{z}$. The decoding of $\vect{z}$ corresponds to $\tildex = D(E(\vect{x}))$. The dimension of $\vect{z}$ is typically bottlenecked to be smaller than the original dimension of $\vect{x}$. This results in the encoder having to define encodings that are as "informative" as possible to reconstruct the original data \cite{bankAutoencoders2021}. $\vect{z}$ may then for instance be used for downstream tasks such as classification or clustering.

\begin{figure}
	\centering
	\includegraphics[width=0.7\linewidth]{autoencoder}
	\caption{Autoencoder neural network architecture adapted from \cite{karagiannakosHowGenerateImages2018}.}
	\label{fig:autoencoder}
\end{figure}


%motivations for VAE
While capable of learning compressed representations, autoencoders do not pose any restrictions on the latent space they define (the space of latent representations). As a result, the representations may be meaningful to computers, but non-interpretable to humans. For instance, given the left image depicted in figure \ref{fig:latent_space_2d} which depicts an autoencoder's two dimensional latent space, it is very difficult to know what the resulting images would be when interpolating between the latent representations of 0 (red) and 1 (blue). Answering this question becomes even more infeasible for higher dimensional latent vectors.

\begin{figure}
	\centering
	\includegraphics[width=0.7\linewidth]{screenshot021}
	\caption{ Both images represent a the two-dimensional latent space, learned from the MNIST dataset \cite{PapersCodeMNIST}. This is a dataset consisting of images of handwritten numbers between 0 and 9. Each image is associated to a class label, referring to the number. The left image consists of the space learned from a classical autoencoder, the right of a variational autoencoder (VAE). Both autoencoders have not received any explicit information of the labels of the dataset, yet learned to define representations which equal numbers are closer to each other. The VAE's latent space, depicted in the right image, is optimised to be standard normally distributed. The latent vectors $\vect{z}$ are distributed according to the two-dimensional normal distribution $\sample{\vect{z}}{\standardnormal}$, where $\bm{\mu}$ is the two dimensional zero vector. $\identitymtx$ is the $2 \times 2$ covariance matrix with ones on the diagonal and zeroes elsewhere.}
	\label{fig:latent_space_2d}
\end{figure}

\subsection{Variational autoencoders}
% notes: 
%1) discuss why, (latent space gaussian)
%2) architecture predicting values
%3) explicit elbo (where q is optimised to be Gaussian) and give intuitions how it squishes all data points together
%4) can be used for latent space, or also 
%4) link with maximising likelihood -> p(x) and how minimises kl divergence between approx and kl divergence

Similar to traditional autoencoders, variational autoencoders (VAE) learn representations that contain the important information that is necessary to reconstruct the data, \textit{however}, an additional constraint is applied to the latent space \cite{doerschTutorialVariationalAutoencoders2021, davidfosterVariationalAutoencoders2023, kingmaAutoEncodingVariationalBayes2022, kingmaIntroductionVariationalAutoencoders2019}. The data points from the latent space are defined in such a way that they conform to a certain distribution, typically, the standard normal distribution $\standardnormal$ \cite{davidfosterVariationalAutoencoders2023}. % ref book oreiley
%The standard normal distribution is a Gaussian distribution (a symmetric distribution characterised by a bell shape), with with a mean of zero and a standard deviation of one \cite{bhandariStandardNormalDistribution2020}. 
This behaviour can be observed in the right plot of figure \ref{fig:latent_space_2d}. Since the latent representations conform a two-dimensional standard normal distribution $\standardnormal$, the data samples are more likely to be near the center at (0, 0).

% TODO: explain about gaps, etc
% i said z's are closer to (0, 0). why do we care? -> similar points are still clustered together, however, there will not be any gaps in the latent space. can thus interpolate between latent representations to find meaningful datapoints.

%o reiley:
%\textit{"A multivariate standard normal distribution is a multivariate distribution with zero valued mean vector and identity covariance matrix."} - 
%%https://learning.oreilly.com/library/view/generative-deep-learning/9781098134174/ch03.html#normal_distribution
%
%\textit{Previously, we saw how there was no requirement for the latent space to be continuous—even if the point (–2, 2) decodes to a well-formed image of a sandal, there was no requirement for (–2.1, 2.1) to look similar. Now, since we are sampling a random point from an area around $z_mean$, the decoder must ensure that all points in the same neighborhood produce very similar images when decoded, so that the reconstruction loss remains small. This is a very nice property that ensures that even when we choose a point in the latent space that has never been seen by the decoder, it is likely to decode to an image that is well formed.} - o reiliy
%





%\textbf{How: predict distributions}

\subsubsection{Simulating distributions through neural networks}
In variational autoencoders the latent representations of a data point $\vecti{x}$ does not simply consist of a fixed deterministic vector $\vecti{z}$, as was the case in the traditional autoencoder. Instead, given $\vecti{x}$, its latent representation corresponds the following distribution $\qphizx$.
A concrete vector $\vecti{z}$ can be obtained taking a sample $\sample{\vecti{z}}{\qphizx}$. As result a single $\vecti{x}$ will correspond to multiple latent vectors $\vecti{z}$. The latent representation $\qphizx$ is modelled as $\normal$. This means that all the required information to model $\qphizx$ is a mean vector $\mui$ and a covariance matrix $\diagsigmai$. Thus, given $\vecti{x}$, a deterministic neural network can simulate $\diagsigmai$ by predicting two vectors, namely, $\mui$ and $\sigmai$. The latent vectors $\vecti{z}$ can then be obtained by randomly sampling from the distribution. And thus, multiple $z^{(i)}$'s may correspond to a single $\vecti{x}$. This method is depicted in figure \ref{fig:vae-repr}. Finally, a sample $\vecti{z}$ can be obtained as follows:

\begin{equation}
\vecti{z} = \mui + \sigmai \odot \epiloni
\end{equation}

where $\epiloni$ corresponds to a sampled value $\samplestandardnormal{\epiloni}$ and $\odot$ is element-wise multiplication. Computing $\vecti{z}$ through $\epiloni$, rather than directly sampling from $\sample{\vecti{z}}{\normal}$ is referred to as the parametrisation trick and allows for gradients to freely backpropagate through the layer \cite{davidfosterVariationalAutoencoders2023}.


% TODO: die sigma(i) moet ^2 zijn!!
\begin{figure}
	\centering
	\includegraphics[width=0.6\linewidth]{"vae repr"}
	\caption{High level view of a variational autoencoder, depicting how a data point $\vecti{x}$ is encoded into a latent distribution and reconstructed as $\tildexi$. Both blocks depict a neural network. The upper block is the encoder and the lower block the decoder. The upper block receives a data points $\vect{x}$ and produces the parameters of $\qphizx$. Since we choose to model $\qphizx$ as a Gaussian with independent components, the covariance matrix $\covariancemtx$ is zero everywhere except for the diagonal. This way the diagonal values, representing the standard deviations, can be represented via a single vector $\sigmai$. The vectors $\mui$ and $\sigmai$ are $\mu(\vecti{x})$ and $\sigma(\vecti{x})$, respectively. These are the output of the encoder block and form the parameters for $\qphizx$. A single neural network with parameter weights $\phi$ is used to simulate $\qphizx$ for every $\vecti{x} \in \mathcal{D} $. This strategy of sharing $\phi$ across data points is refered to as "amortised variational inference" \cite{kingmaIntroductionVariationalAutoencoders2019}.}
	\label{fig:vae-repr}
\end{figure}

%$$\vecti{x} \in \mathcal{D} $$
%$$\vect{z}$$
%$$\mu(\vect{z})$$
%$$\sigma(\vect{z})$$
%
%$$\mu(\vecti{x})$$
%$$\sigma(\vecti{x})$$
%
%
%$$\sample{\vect{z}}{\normal}$$
%
%
%$$ \qphizx $$
%$$ \pthetaxz $$
%
%$$ \widetilde{\mathbf{x}} ^ {(i)} $$

\subsubsection{The learning objective}

So far we have discussed how predictions of a neural network can emulate predicting a distribution $\qphizx$, by predicting the distribution's parameters $\mui$ and $\sigmai$. However, no constraints have been set on the quality of the distributions. We will discuss this now. The representations are optimised to minimise two measurements: one, the reconstruction error, and secondly, the distance from the latent distributions to $\standardnormal$.

The loss function to be optimised for a single data point $\vecti{x}$ is shown in the equation below.

\begin{equation}
	\mathcal{L} = \elboexplicit \label{eq:elbo_explicit_intial}
\end{equation} % src: ref to slides ugent

% 1) reconstruction error
Although the equation may seem daunting at first, we will decompose its components. The loss function is made up of two terms, the left term corresponds to the reconstruction error, while the second term poses constraints on the latent space. Let us focus on the reconstruction term first.

$$
\reconstr
$$

The distribution $\pthetaxz$ is a distribution over $\vecti{x}$ where $\vect{z}$ is instantiated. What comes out are thus probabilities. Notice $\vect{z}$ is sampled from the Gaussian distribution $\sample{z}{\qphizx}$. So the $\vect{z}$'s close to the mean are more likely to be sampled than the onces further away. So for these $\vect{z}$'s, we'd like their probability of corresponding to an actual $\vect{x}$ to be high as well. Adding a log to this equation does not change the parameters to obtain a maximum. Finally, by adding a negative sign in front, maximising the equation is equivalent to minimising the negative. % todo: arg is not entirely clear
%todo: look into max like estim, as there hopefully i can find a meaningful defintion for "maximising..."
In practice this term is approximated through mini-batches with the mean squared error.


Optimising the second term of equation \ref{eq:elbo_explicit_intial} poses the constraints on the latent distributions $\qphizx$. Again, this metric should be minimised. As we discussed in the chapter on Entropy, the KL divergence can be considered as a (non-symmetric) distance measure between two distributions. Hence, this value is small when the two distributions are similar. As we discussed earlier $\pz$ is often replaced by $\standardnormal$ as shown below. Minimising this equation will result in moving each distribution $\qphizx$, corresponding to a value $\vecti{x}$, close to $\standardnormal$.

$$
\latentspaceconstraintstandardgaussian
$$

One can algebraically prove, that optimising this equation is equivalent to optimising the following equation \cite{kingmaAutoEncodingVariationalBayes2022}: 

\begin{equation}	
	\kl{\normal}{\standardnormal} = \latentspaceconstraintclosedform
\end{equation}

where $\sigma_k^{(i)}$ and $\mu_k^{(i)}$ correspond to the components of the predicted vectors $\sigmai$ and $\mui$, respectively.


% 2) kl divergence

\subsubsection{[TODO] Relation with variational inference / derivation of ELBO loss}

%A mapping function $f_W(x) = y$ can be predict distributions by predicting the parameters of the the distribution (mu, sigma for Gaussian distributions).
%
%Rather than directly learning parameters $\phi$ that map a data point $\vecti{x}$ to latent distribution $p(\vect{z} \mid \mathbf{x}^{(i)})$, an \textit{approximate distribution} $\qzx$ is learned that approximates $p(\vect{z} \mid \vecti{x})$. Through gradient descent parameters for $\phi$ can be achieved, by minimising the KL divergence between the two distributions. $\condq{\vect{z}}{\vecti{x}}$ will is expressed as Gaussian distribution.
%
%
%\begin{equation}
%	\kl{\qprobzx}{\probzx} \label{eq:kl_q_p}
%\end{equation}
%
%Yet, the idea is nice, optimising $\qprobzx$ to approximate $\probzx$ sounds nice, it is only possible when $\probzx$ is known, which it is not. If it was, there was no need to approximate it.
%
%Equation \ref{eq:kl_q_p} is thus intractable to compute, yet the following is true:
%
%\begin{equation}
%	\begin{aligned}
%		\kl{\qprobzx}{\probzx} & = \expectedsamplezq \left[ \log \frac{\qzx}{\pzx} \right] \\
%		& = ... (todo) \\
%		& = - \left( \elbo \right) + \logevidence
%	\end{aligned}
%\end{equation}
%
%By replacing the terms, we obtain the following equation:
%
%\begin{equation}
%	\begin{aligned}
%		\logevidence & = 
%		\klapproxposterior + \elbo \\
%		& = \klapproxposterior  + \lelbo
%	\end{aligned}
%\end{equation}
%
%We can observe that the KL divergence between the true and approximate posteriors, $p$ and $q$ respectively, is bounded by $\logevidence$. Hence, although $\pzx$ is unknown, maximising $\lelbo$ results in minimising the KL divergence. Also notice that $\kl{\qprobzx}{\probzx} >= 0$, and thus the maximum value for $\lelbo$ results in a KL divergence between the posteriors of zero. It thus suffices to maximise $\lelbo$, or equivalently minimise $-\lelbo$. The loss for a single data point $\vecti{x}$ then becomes

%TODO \textbf{todo: in paper variational bayes, they refer to p(x) as p theta, but in slides ugent not?}. \textbf{todo: why cant p(x) be very big then? probably has something to do with jensens inequality.}




\begin{equation}
	\begin{aligned}
		- \lelbo & = - \elbo \\  
		& = \elboexplicit \label{eq:elboexplicit} \\
	\end{aligned}	
\end{equation} % src: ref to slides ugent

Where the first term is reconstruction error and second is regularisation. Hence minimising the KL divergence between the two posteriors is equivalent to minimising the divergence between the approximate posterior $\qzx$ and the \textbf{marginal or prior?} $\pz$.

$p(z)$ in the equation is usually chosen to be the standard normal distribution $\mathcal{N}(0, I)$, such that a closed form solution exists. As such, when $p(z)$ corresponds to the standard normal, and $\qzx$ is a multidimensional Gaussian with mean vector $\mui$ and covariance matrix with independent dimensions, such that the diagonal corresponds of a vector standard devisations $\sigmai$, then the KL divergence has the following closed form:

\begin{equation}	
	\kl{\normal}{\standardnormal} = \latentspaceconstraintclosedform
\end{equation}

For the equation above, gradients can easily be back propagated through machine learning libraries such as Tensor Flow. We still require a method for computing the gradient of the first term in $\lelbo$. % todo: gradient van linker kan door sampling en mini batch.











%Recent Advances in Autoencoder-Based:
%Representation Learning https://arxiv.org/pdf/1812.05069.pdf
%they speak about disentanglement of vae and indepndent features

%Understanding disentangling in β-VAE
%https://arxiv.org/pdf/1804.03599.pdf
%"β-VAE aligns latent dimensions with components that make different contributions to reconstruction"
%Our key hypothesis is that β-VAE finds latent components which make different contributions to the log-likelihood term of the cost function (Eq. 5). These latent components tend to correspond to features in the data that are intuitively qualitatively different, and therefore may align with the generative factors in the data.











%- \subsubsection{reparametrisation trick}
%o reiley:
%\textit{THE REPARAMETERIZATION TRICK
%	Rather than sample directly from a normal distribution with parameters $z_mean$ and $z_log_var$, we instead sample epsilon from a standard normal and then manually adjust the sample to have the correct mean and variance.}
%\textit{This is known as the reparameterization trick and is important as it means gradients can backpropagate freely through the layer. By keeping all of the randomness of the layer contained within the variable epsilon, the partial derivative of the layer output with respect to its input can be shown to be deterministic (i.e. independent of the random epsilon), which is essential for backpropagation through the layer to be possible.}

%\subsubsection{the loss function - from oreiley}
%The Loss Function
%Previously, our loss function only consisted of the reconstruction loss between images and their attempted copy after being passed through the encoder and decoder. The reconstruction loss also appears in a variational autoencoder, but we require one extra component: the Kullback–Leibler (KL) divergence term.
%
%KL divergence is a way of measuring how much one probability distribution differs from another. In a VAE, we want to measure how much our normal distribution with parameters z_mean and z_log_var differs from a standard normal distribution. In this special case, it can be shown that the KL divergence has the following closed form:
%
%kl_loss = -0.5 * sum(1 + z_log_var - z_mean ^ 2 - exp(z_log_var))
%or in mathematical notation:
%
%The sum is taken over all the dimensions in the latent space. kl_loss is minimized to 0 when z_mean = 0 and z_log_var = 0 for all dimensions. As these two terms start to differ from 0, kl_loss increases.
%
%In summary, the KL divergence term penalizes the network for encoding observations to z_mean and z_log_var variables that differ significantly from the parameters of a standard normal distribution, namely z_mean = 0 and z_log_var = 0.
%
%Why does this addition to the loss function help?
%
%Firstly, we now have a well-defined distribution that we can use for choosing points in the latent space—the standard normal distribution. Secondly, since this term tries to force all encoded distributions toward the standard normal distribution, there is less chance that large gaps will form between point clusters. Instead, the encoder will try to use the space around the origin symmetrically and efficiently.
%
%In the original VAE paper, the loss function for a VAE was simply the addition of the reconstruction loss and the KL divergence loss term. A variant on this (the 
%-VAE) includes a factor that weights the KL divergence to ensure that it is well balanced with the reconstruction loss. If we weight the reconstruction loss too heavily, the KL loss will not have the desired regulatory effect and we will see the same problems that we experienced with the plain autoencoder. If the KL divergence term is weighted too heavily, the KL divergence loss will dominate and the reconstructed images will be poor. This weighting term is one of the parameters to tune when you’re training your VAE.



%\subsubsection{Disentangled latents + posterior collapse}


\section{Representation learning through Noise-contrastive estimation}




% FEEDBACK: MORE EXPLICITLY STATE MAIN IDEA OF REPR LEARNING AND HOW CPC FITS INTO THIS FRAMEWORK
\subsection{Contrastive predictive coding}


%\textbf{[What: CPC: repr for sequences]} \\
	In what follows next, we discuss Contrastive Predictive Coding (CPC), a representation learning approach that we use as the basis for our own experiment in the following chapter.
	% TODO: cpc differs from autoencoders in two key ways: reconstruct + time series data
	CPC is an unsupervised learning approach, again with the objective of learning (lower dimensional) representations from high dimensional data \cite{oordRepresentationLearningContrastive2019}. While the objective is thus the same as for the autoencoders discussed in the previous section, CPC achieves its representations entirely differently. An autoencoder's objective is to define a compressed representation from which the original data can be recovered. However, when working with sequential data, compressing patches of the sequence without considering the relation with nearby patches, will result in lost information as the context between patches is not encoded into the representation. % todo: is based on blog, but i cant really refer.. https://www.lesswrong.com/posts/XE6LD2c9NtB7gMdEm/an-92-learning-good-representations-with-contrastive. I should however give an example.


\begin{figure}[h] % cpc overview
	\centering
	\includegraphics[width=0.7\linewidth]{"cpc overview"}
	\caption{Overview of Contrastive Predictive Coding, originates from \cite{oordRepresentationLearningContrastive2019}}
	\label{fig:cpc-overview}
\end{figure}

%\textbf{[maintain information between patches by predicting the future based on the past? + architec fig + var names]} \\
	% todo
	CPC deals with these context issues by maximising the shared information between the extracted representations of temporally nearby patches \cite{lowePuttingEndEndtoEnd2020}. We will discuss this concept in more detail in a following section. For now, we would like to draw the readers attention to figure \ref{fig:cpc-overview}. The figure depicts a high level view of how this idea is achieved by producing two latent latent representations $\vect{z}_i$ and $\vect{c}_i$. An audio sequence of undefined length is split up into patches $\x_1...\x_n$ where each $\x_i$ is a vector of fixed length, containing for instance 10ms of speech audio. Each patch $\x_i$ is encoded into latent representation $\zt$, defined as follows:
	
	$$
	\zt = g_{enc}(\xt) .
	$$
	
	$g_{enc}( . )$ is for instance a convolutional fully connected ANN. The latent representations $\z_1..\z_n$ are obtained independently from each other and do not yet contain any information on an historic context. This historic context is achieved through $g_{ar}( . )$, an auto-regressor, which encodes all previous $\z_1 ... \zt$ into a single representation $\ct$:
	
	$$
	\ct = g_{ar}(\z_1 ... \zt)
	$$
	
	Either $\zt$ or $\ct$ could be used as latent representation for downstream tasks. Oord et al. suggest to use $\ct$ for tasks where context about the past is useful, for instance speech recognition, and $\zt$ when historic context is not useful \cite{oordRepresentationLearningContrastive2019}. As shown in figure \ref{fig:cpc-overview}, the encodings from sequential data of undefined length, may correspond a series of latent representations $\vect{c}_1, \vect{c}_2, ... $ or $\z_1, \z_2, ... $. In the case of downstream tasks which require a single representation vector, Oord et al. propose to pool the sequence of vector representations into a single vector.
	

\subsubsection{Slowly varying features}
%\textbf{[assume slow features. + why maxim mutual info? between past and future]}\\
	The temporally nearby patches $\z_{t+1}$ and $\ct$ are optimised to preserve shared information, while discarding differences. Before we discuss how to obtain such representations, we first motivate why defining representations in this fashion makes sense.
	
	Consider we would like to define useful representations for sequential data such as speech signals. Then it not unlikely to believe that the conveyed information at time step $t$ and $t+k$ contains some redundant information, such as pitch, frequency, tone, etc. \cite{raoUnderstandingGradientIsolatedLearning2020}. Meanwhile, large changes of the signal in a small time window, may be the result of noise. Sequential data which poses these slowly varying features, are commonly referred to as "slow-features" \cite{zhangSlowFeatureAnalysis2012}. CPC leverages these slowly varying features, by encoding the underlying shared information between different patches, while at the same time discarding low-level information and noise that is more local \cite{oordRepresentationLearningContrastive2019}.


\subsubsection{The learning objective}
%\textbf{[discriminate pos from neg samples + formal MI]}\\
	% the latent representations are optimised through a similarity function f. It will turn on that optimising the similarity between two representations is equivalent to maximising their mutual information.
	
	
	CPC will learn to preserve information between temporally nearby representations, by solving another task. In particular, CPC learns to discriminate subsequent \textit{positive} samples $\ztk$ from \textit{negative} random samples $\zj$. This is achieved through a similarity function $f_k(.)$, which scores the similarity between two latent representations \cite{lowePuttingEndEndtoEnd2020}. It is defined as a log bilinear model as follows:
	\begin{equation} % f_k
		\fkdefinition \label{eq:fk}
	\end{equation}
	where $W_k$ is a weight matrix which is learned. $f_k( \zj , \ct )$ thus quantifies how likely the context $\ct$ corresponds to a random vector $\zj$. Due to the slowly varying data assumption, a good representation for successive representations $\ztk$ and $\ct$ is one where $f_k( \ztone, \ctone)$ is high and $f_k( \zj, \ct)$ is small for random $\zj$. Or equivalently, maximising the shared information between temporally nearby patches, while discarding the temporal noise results in large values $f_k( \ztone, \ctone)$.
	%Had er miss ergens bij gekund: Finding the correct $\ztone$ in a batch of random $zjs$, given $\ct$ is thus equivalent to "predicting the future given the past" (is related to predictive coding).
	
	%	CPC learns to discriminate subsequent 'positive' samples $x_{t+1}$ from 'negative' random samples $\x_j$, hence the name “contrastive noise estimation”. CPC exploits this so called ‘slow-features’ property, by encoding the common information between nearby parts of the signal, while also ignoring the noise which is more local. It will turn out that maximising this shared information, corresponds the notion of mutual information, we discussed earlier. We discuss this more in more detail in a following subsection.
	%	\begin{equation} % MI
	%		I(z; c) = \sum_{z, c} p(z, c) \log \frac{p(z \mid c)}{p(x)}
	%	\end{equation}
	%	Hence CPC achieves latent representations $\z_{t+1}$, by defining them in such a way that the mutual information between the past $ct$ and the future $\x_{t+1}$ is maximised.

	%To recognise that $f_k$ does indeed quantify the similarity between the two latent representations, consider the case where we are given $f_k$ with optimal weights $W_k$. Note that $W_k c_t$ results in a vector. Since, the dot product of two vectors is large when the vectors point in similar directions and negative in opposite directions, $f_k$ is large when $z_j$ and $c_t$ have a lot of similarity. Meanwhile the value is low when there is little information in common. Also notice the $\exp$ in the equation, this prevents negative values, which will turn out useful when $f_k$ is inserted in the NCE loss function $\mathcal{L}_n$, which we below. 

%[L]
	%Note that in the case of a one layered neural network, $\zt$ corresponds to the vector $\sigma(x_t^T W)$, where W is a weight matrix to be optimized (different from $W_k$ in equation \ref{eq:fk}). Hence, CPC must optimize, at least two weight matrices, namely W and the Wk in equation \ref{eq:fk}. We now discuss how the neural network obtains these optimal weights. Just like in supervised ANNs, a loss function must be optimized. Instead of optimising the generic loss function mentioned in equation ref, the following loss InfoNCE loss function is minimized, which will result in the optimized function $f_k$.
	
	The InfoNCE loss, used to optimise $g_{enc}$, $g_{ar}$ and $W_k$ simultaneously is shown below. 

	\begin{equation} % L_N
		\mathcal{L}_N = - \sum_{k} \expected{\textsubscript{X}} \left[ \log \nceprediction \right] \label{eq:NCE_loss}
	\end{equation}
	
	where $X$ corresponds to the set ${ \left\{ \ztk, \z_1, \z_2, ... \right\} }$. Notice that there exists exactly one $\ztk \in X$, which corresponds to a positive sample and all other $\zj \in X$ correspond to negative samples (with respect to the context $\ct$).  Hence good representations for $\z, \vect{c}$ and $f_k(.)$, will result in a large similarity score for positive samples and approximate $0$ for negative samples, resulting in a minimum fraction equal to $1$. This would then be cancelled out by the $\log(.)$ function. Meanwhile, $\mathcal{L}_n$ is large when the denomitator is large, indicating in a large high similarity score for negative samples.
	
\subsubsection{Ties with mutual information}
	Earlier we argued that CPC's encodings will preserve shared information between temporally nearby patches, while discarding the local noise. Oord et al. make this claim even stronger by making ties with mutual information, which we discussed in a previous chapter. In particular, Oord et al. proof that optimising InfoNCE is equivalent to maximising the mutual information between $\ct$ and $\ztone$ \cite{oordRepresentationLearningContrastive2019}. 
	
	\begin{equation} % MI
		I(\ztone; \ct) = \sum_{\ztone, \ct} p(\ztone, \ct) \log \frac{p(\ztone \mid \ctone)}{p(\ztone)}
	\end{equation}
	
	This proof is available in their appendix. Although, we do not repeat the proof here, we give a high level overview.
	
	The first step in proving the relation between the InfoceNCE loss and mutual information is to model $\fkzc$ in a probabilistic manner. The InfoNCE loss is in fact the categorical cross-entropy of classifying the positive sample correctly with $\frac{f_k}{\sum_{X} f_k}$ as the predicted model \cite{oordRepresentationLearningContrastive2019}. Since this equation may take values between zero and one, it can be considered as a probability. In particular, the optimal probability for the loss can then be written as 

	$$p(i \mid X, \ct)$$
	
	where $X$ corresponds the set of samples  ${ \left\{ \ztk, \z_1, \z_2, ... \right\} }$  as discussed in the InfoNCE loss, and $i$ corresponds to indicator that sample $\zi$ is the "positive" sample. By doing so, one can eventually obtain a proportionality relation to the density distribution presented below. 
	
	\begin{equation}
		\fzc \propto \pzcdivpz \label{eq:fkproporational}
	\end{equation}
	
	% onjuist?
	%This equation can be interpreted as follows: there exists a constant $r \in \R^+$ such that $r \times \fzc = \pzcdivpz$. Indeed, when the similarity between $x_{t+k}$ and $c_t$ is large, $f_k$ is large, and thus also $p(z_{t+k}|c_t)$, which is normalized by $\pztk$.
	Oord et al. utilise this proportionality relation to reformulate $-\mathcal{L}_N$ as a lower bound on the mutual information between $\ztone$ and $\ct$ as follows \cite{lowePuttingEndEndtoEnd2020, oordRepresentationLearningContrastive2019}:
	
	\begin{equation}
		I(\ztone; \ct) \ge \log(N) - \mathcal{L}_N
	\end{equation}

	Since number of samples $N$ is a constant, the mutual information between $\ztone$ and $\ct$ becomes greater when $\mathcal{L}_N$ becomes smaller. Additionally, when the number of samples $N$ increases, the bound becomes tighter.
	
			

	\textbf{TODO: if time permits: compare against VAE, so doesn't have a decoder layer so simpler architecture + allows for sequential data.}
	
		% rnd quote
		%\textit{By using a contrastive loss, high-dimensional representations of subsets of each data point are used for predicting the future subsets of the same sample. } - % https://web.archive.org/web/20220616074256id_/http://proceedings.mlr.press/v136/stacke20a/stacke20a.pdf
	

\subsection{Greedy InfoMax}
	% successes
	So far we discussed how CPC encodes latent representations by maximising the mutual information between temporally nearby patches of data. This method has shown great success in recent years and is considered state of the art in self-supervised learning for encoding sequential data \cite{stackeEvaluationContrastivePredictive2020}. Additionally, CPC has been successfully applied to multiple use cases \cite{stackeEvaluationContrastivePredictive2020, dehaanContrastivePredictiveCoding2021, luSemiSupervisedHistologyClassification2019, bhatiSegmentalContrastivePredictive2021b, deldariTimeSeriesChange2021, henaffDataEfficientImageRecognition2020}. This is achieved by minimising the InfoNCE loss discussed earlier in equation \ref{eq:NCE_loss}. Through this \textit{global} loss function all parameters are optimised end-to-end via backpropagation. Although, backpropagation is considered effective, it still suffers from multiple constraints. Löwe et al. group the constraints in two categories: biological and computational.
	

	\subsubsection{Biological constraints of backpropagation [TODO]}
		- \textbf{Local Error Representation, Weight Symmetry, ...}
		its performance is still uncompanionable to 
		
		% vs humans learning
		When comparing its performance against how humans learn, humans can still learn much faster from only a few examples... 
		
		The human brain does not appear to have a global objective function, which is optimized by backpropagating an error signal \cite{marblestoneIntegrationDeepLearning2016}. In the following section we discuss the changes Löwe et al. have made to make CPC more biologically plausible.
		
	\subsubsection{Computational constraints of backpropagation [TODO]}
		- \textbf{- memory overhead, synchronous training, vanishing gradients.}
		
	
	%[same ideas of repr learning via mutual information maximising, of latent representations, diff: GIM = Biological approach]
	Löwe et al. work further on the representation learning method proposed in CPC by taking a biologically inspired approach. This is done via the introduction of Greedy InfoMax (GIM), an extension on CPC. As in CPC, mutual information between latent representations is maximized by optimising the InfoNCE loss function. However, neurons in the biological brain primarily learn from local information and there does not appear to be a global error function that is optimized \cite{caporaleSpikeTimingdependentPlasticity2008}. Hence, end-to-end backpropagation cannot be used as an accurate learning model for the brain. 



	\begin{figure}[h!t]
	\hspace{2cm}
	\begin{annotatedFigure}
		{\includegraphics[width=0.8\linewidth]{"GIM overview"}}
		
		\annotatedFigureText{0.282,0.99}{black}{0.31}{$\mathcal{L}_N^1$}{8}
		\annotatedFigureText{0.61,0.99}{black}{0.31}{$\mathcal{L}_N^2$}{8}
		\annotatedFigureText{0.82,0.99}{black}{0.31}{$\mathcal{L}_N^M$}{8}
		
		
		\annotatedFigureText{0.282,0.77}{black}{0.31}{$\mathcal{L}_N^1$}{5}
		
		\annotatedFigureText{0.145,0.85}{black}{0.31}{$f_1$}{5}
		\annotatedFigureText{0.145,0.7}{black}{0.31}{$f_1$}{5}
		
		\annotatedFigureText{0.24,0.68}{black}{0.31}{Positive sample}{5}
		\annotatedFigureText{0.18,0.88}{black}{0.31}{Negative sample}{5}
		
		
		\annotatedFigureText{0.143,0.55}{black}{0.31}{$\ct$}{5}
		\annotatedFigureText{0.18,0.545}{black}{0.31}{$\ctone$}{5}
		\annotatedFigureText{0.23,0.545}{black}{0.31}{$\vect{c}_{t+2}$}{5}
		
		\annotatedFigureText{0.143,0.44}{black}{0.31}{ar}{5}
		\annotatedFigureText{0.195,0.44}{black}{0.31}{ar}{5}
		\annotatedFigureText{0.4,0.44}{black}{0.31}{ar}{5}
		
		
		\annotatedFigureText{0.143,0.3}{black}{0.31}{$\zt$}{5}
		\annotatedFigureText{0.18,0.295}{black}{0.31}{$\ztone$}{5}
		\annotatedFigureText{0.23,0.295}{black}{0.31}{$\z_{t+2}$}{5}	
		
		\annotatedFigureText{0.09,0.175}{black}{0.31}{enc}{5}
		\annotatedFigureText{0.161,0.175}{black}{0.31}{enc}{5}
		\annotatedFigureText{0.399,0.175}{black}{0.31}{enc}{5}
		
		\annotatedFigureText{0.1,0.05}{black}{0.31}{$\xt$}{5}
		\annotatedFigureText{0.155,0.045}{black}{0.31}{$\xtone$}{5}
		\annotatedFigureText{0.22,0.045}{black}{0.31}{$\x_{t+2}$}{5}
		
		
%		\annotatedFigureText{0.59,0.01}{black}{0.31}{$\vect{c}^1$}{8}
%		\annotatedFigureText{0.79,0.01}{black}{0.31}{$\vect{c}^{M-1}$}{8}
		\annotatedFigureText{0.6,0.01}{black}{0.31}{$\vect{c}_1^1, \vect{c}_2^1, ...$}{8}
		\annotatedFigureText{0.8,0.004}{black}{0.31}{$\vect{c}_1^{M-1}, \vect{c}_2^{M-1}, ...$}{8}
		
		% \annotatedFigureText{0.49,0.57}{black}{0.31}{$\vect{c}^1$}{8}
		% \annotatedFigureText{0.45,0.31}{black}{0.31}{$\z^1$}{8}
		
		\annotatedFigureText{0.9,0.55}{black}{0.31}{$\vect{c}_1^M, \vect{c}_2^M, ...$}{8}
		
		
	\end{annotatedFigure}
	\caption{Overview of Greedy InfoMax}
	\label{fig:gim-overview}
	% Dieter uses this: https://www.overleaf.com/learn/latex/TikZ_package
\end{figure}


	

\subsubsection{Towards greedy learning}
	%[split up into layers]
	Instead of optimising a single loss function (as in end-to-end backpropagation), Löwe et al. split the neural network up into "modules". Each module is optimised with its own personal instance of the InfoNCE Loss function $\mathcal{L_N}$ described in equation \ref{eq:NCE_loss}. This idea is depicted in figure \ref{fig:gim-overview}, which displays $M$ modules, each trained with their own instance of the InfoNCE loss. Each "module" can be considered as an instance of the CPC framework where high dimensional data goes in and a lower dimensional representation comes out. The output representations of a one module serve as the input for the next module. One can thus think of Greedy InfoMax as "stacking" multiple ANN's, which were each trained with the InfoNCE loss on top of each other. A module may consist of one or more layers of the neural network. 
	
\subsubsection{Optimising Greedy InfoMax}

% during training, the outputs of one module serve as the inputs for the next module, the latent representations for a module m, can written as follows



% modules are can be trained asynchrounsly, and predictions flow through



By splitting up the neural network into modules and allowing them to optimise their own loss, modules can be trained asynchronously, improving upon training time when multiple GPUs are used. Note that similar to tradition neural networks, during forward propagation each module (or layer) requires the output of the previous layer as its input. Hence, modules are still dependent on previous modules for training, however, they can learn without the gradients of subsequent modules.

	
	We now revisit the function $f_k$ and NCE Loss, and describe it in the context of GIM. Instead of a single $f_k$ and $L_N$, we now have a function $f_k^m$ for each module m, each with their own respective loss $\mathcal{L_N}^m$. The attentive reader may notice $f_k^m( . )$ no longer receives as input $c_t$, but instead a second $z_t$. Löwe omits the autoregressive function \textbf{TODO: F(X) = CT OF ZO} to obtain $c_t$ and immediately applies  gives opted for this function instead, as she did not notice any improvement in her preliminary results.
	
	$$
	f_k^m(\ztk^m,\zt^m) = \exp({\ztk^m}^TW_k^m\zt^m)
	$$
	
	$$
	\mathcal{L}_N^m = -\sum_{k} \expected{\textsubscript{X}} \left[\log \frac{\fkm }{\sum_{\zj^m \in X} f_k^m(\zj^m, \zt^m)} \right]
	$$
















%************************ BS:
%Consider a neural network that maps input data $x$ to output $y$.
%More formally one may write 
%$$ f_\theta (x^{(i)}) = y^{(i)} $$
%where y represents the class and $\theta$ the model parameters. However, this mapping can also be written in a probabilistic manner as follows
%%$$ p_\theta( y^{(i)} | x^{(i)}) $$
%%where p represents a probability densinity function.
%%Hence $f_\theta$ can be rewritten as
%%$$ f_\theta (x^{(i)}) = argmax_{\textsubscript{y $\in \mathcal{ Y }$}} p_\theta( y \mid x^{(i)}) $$
%
%
%$$ p_\theta( y^{(i)} | x^{(i)}) $$
%where p represents a probability densinity function.
%Hence $f_\theta$ can be rewritten as
%
%$$
%f_\theta (x^{(i)}) =
%\underset{\textsubscript{y $\in \mathcal{ Y }$}}{\mathrm{argmax}} \left[ p_\theta( y \mid x^{(i)}) \right] 
%$$

%************************ BS



\chapter{Variational Greedy InfoMax}

\section{Motivation} %\section{Problem setting}
	In the previous section we discussed two categories of representation learning though deep learning. First, we discussed the autoencoder and its variational counterpart, which minimise the reconstruction error. Secondly, we discussed Contrastive Predictive Coding and Greedy InfoMax, both of which optimise the Info NCE objective. This category seeks to maximise the mutual information between the encodings of data patches that are temporally nearby. The latent representations obtained from all four methods can then be utilised for downstream tasks \cite{bengioRepresentationLearningReview2013, weiRecentAdvancesVariational2021, oordRepresentationLearningContrastive2019, lowePuttingEndEndtoEnd2020}
	
	% repr learn autoenc + vae (disentenglement)
		The autoencoder's sole objective is to define representations to reconstruct the original data. As a result, the representations may serve well for data compression, however, no additional constraints are enforced, such as feature disentanglement and thus the latent space may still be hard to work with for downstream tasks \cite{tschannenRecentAdvancesAutoencoderBased2018}. Meanwhile, VAEs' additional regularisation term, results in representations which break down or disentangle each feature into a narrowly defined variable and encodes them as separate dimensions \cite{weiRecentAdvancesVariational2021}. This additional constrained may result in better suited representations for downstream tasks. % TODO: I could reformulate this, and mention meta priors such as in this paper: https://arxiv.org/pdf/1812.05069.pdf

	% cpc contrasts noise -> smaller architect
		Both autoencoders and VAEs merely learn to reconstruct the data. Hence, all the "information" that is important to reconstruct the data will be maintained in the latent representation, whether the information is useful for the downstream task or not. Meanwhile, optimising latent representations for the InfoNCE objective will maintain shared information between temporally nearby patches, while discarding local noise. Reconstruction is thus not needed for training. This strategy has the tremendous benefit that a decoder block is not required, resulting in a significantly simplified architecture, meanwhile maintaining state-of-the-art performance \cite{stackeEvaluationContrastivePredictive2020}. A second benefit of these mutual information maximisation models is that they are directly compatible with sequential data.
		
	% lead to interpretabil
		Both categories (reconstruction and information maximisation algorithms) possess the ability to obtain useful representations for various downstream tasks. However, the content of these representations may not always be intuitive to humans and their structure may be difficult to comprehend. While CPC and GIM are considered state-of-the-art, their performance comes at a cost of having the least interpretable representations. Autoencoders maintain interpretability by using a decoder to reveal the information contained in the latent representation. The same transparency can also be achieved with VAEs. Additionally, by using a standard Gaussian as a prior and constraining the latent distributions to be similar to this prior, we can interpolate between representations and observe the effects through the decoder. As such, we can observe the specific information that is contained in each of the representation's features. VAEs can also result in disentangled features, further enhancing interpretability \cite{grossuttiDeepLearningInfrared2022}. In contrast, CPC and GIM do not contain a built in decoder mechanism, nor pose constraints on the latent space, significantly reducing interpretability.
		


\section{Towards decoupled training for probabilistic representations}
	% Our contribution
		In what follows next we introduce Variational Greedy InfoMax (V-GIM), maintaining the state-of-the-art performance obtained from optimising InfoNCE, while leveraging the interpretable and disentangled benefits from VAEs. This is achieved by optimising a novel loss function, \textit{Variational-InfoNCE}, a combination of InfoNCE and the regularisation term from VAEs. Additionally, by splitting up the neural network into modules, as introduced in \cite{lowePuttingEndEndtoEnd2020}, we greedily optimise each module with its own instance of this loss function. As a result, the interpretability benefits from VAEs will also be applicable in-between modules. This is in contrast to VAEs where solely the final output representations are interpretable.		
				
	% How:
		% still maximise mutual information between zt, ztk, but predictions no longer fixed datapoints.
		% xt -> cpc model -> q( . | xt) = mui, sigmai
		
		As discussed in the section on Contrastive Predictive Coding (CPC), a patch of sequential data $\xt$ is encoded through $g_{enc}(\xt) = \zt$ and aggregated over previous encodings through auto-regressor $g_{ar}(\z_1  \dots \zt) = \ct$, where both $\zt$ or $\ct$ may serve as representations for downstream tasks. The encoder function $\genc$ is represented as neural network, eg via a CNN, and $\gar$ for instance as a GRU. % todo: are gru and cnn used before?
		Finally, the encoding functions $\genc$ and $\gar$ are obtained by optimising a global loss function, the InfoNCE loss, end-to-end via backpropagation. 

		% split in modules
			Instead, in this study, we split up $\genc$'s network architecture by depth into $M$ modules 
			$$g_{enc}^1(\cdot),~ g_{enc}^2(\cdot),~\dots,~g_{enc}^M(\cdot)$$ 
			and prevent gradients from flowing between modules, as introduced in \cite{lowePuttingEndEndtoEnd2020}. An additional optional $M+1$'th module $\gar$ can be appended to the architecture. Each module is greedily optimised via a novel loss function, $\Lvnce$, which we will define in a following subsection. Each module's output serves as input for the successive module, as presented in the following equations, and depicted in figure \ref{fig:variationalgim}.
			
			\begin{figure} % fig: overview multiple modules
				\centering
				\includegraphics[width=0.7\linewidth]{temp_variational_gim}
				\caption{}
				\label{fig:variationalgim}
			\end{figure}
			
			\begin{align*} % g_enc1, ...
				g_{enc}^1(\xt) &= \zt^1 \\
				g_{enc}^m(\zt^{m-1}) &= \zt^m \\
				g_{ar}(\z_1^M ~ \dots ~ \zt^M) &= \ct
			\end{align*}
			
			The final representation $\ct$ is obtained by propagating $\xt$ through each modules as follows:
			$$ g_{ar}(g_{enc}^M ( \dots	g_{enc}^2(g_{enc}^1(\xt)))) $$
			
			% TODO: WANNEER OVER GRADIENTS BEGINT, A SINGLE MODULE IS DEFINED AS FOLLOWS.. MET F(z m-1) -> (mu, sigm)
			
			
					
		% Distributions
			Additionally, taking inspriation from VAEs, the outputs from $\gencm$ and $\gar$ are in fact samples from a distribution denoted by $q(\zt^m \mid \zt^{m-1})$, defined as a multivariate Gaussian with diagonal covariance matrix, as follows:
			$$\sampleqdot{\zt^{m-1}} = \normalfatmusigma$$
			with $\mufat$ and $\sigmafat$ dependent on $\zt^{m-1}$, specified in more detail in a following subsection.
			The outputs for $\gencm$ and $\gar$ are obtained by sampling from this distribution, denoted respectively, as follows:
 			\begin{align} % z ~ q AND c ~ q
			 	\sample{\zt^m}~ & \qfromzmneg  \label{eq:sample_z_from_q} \\
			 	\sample{\ct}~ & \qfromzM
			 \end{align}
			Modules are thus stochastic and computing $g_{enc}^m(\zt^{m-1})$ twice will likely result in two different representations of $\zt^m$. This is in contrast to CPC and GIM's encodings which remain fixed depending to the input \cite{oordRepresentationLearningContrastive2019, lowePuttingEndEndtoEnd2020}.
		
		% How distributions: predict q + sample
			We achieve these stochastic modules by defining each module $\gencm$ consisting of two blocks. The first block receives as input $\zt^{m-1}$ and predicts the parameters $\mufat$ and $\sigmafat$. These two parameters describe the distribution $\qfromzmneg$. Since we defined $q$ as Gaussian with a diagonal covariance matrix, the distribution can be fully described by those two vectors. The second block samples 
			$\sample{\zt^m} \sampleqdot{\zt^{m-1}}$ from this distribution and produces an output representation. This is depicted in figure \ref{fig:single_variational_module}.
			
			\begin{figure}[h] % img: single module
				\centering
				\includegraphics[width=0.7\linewidth]{temp_variational_module}
				\caption{}
				\label{fig:single_variational_module}
			\end{figure}
			
		% define ztm~q() = gaussian., --> z = mu + sigma*noise
			In practice, sampling from $q$ is achieved through a reparametrisation trick, as introduced in \cite{kingmaAutoEncodingVariationalBayes2022}. The equation to compute $\zt^m$ then becomes:
			\begin{equation*}
				\zt = \mufat + \sigmafat \odot \epilonfat
			\end{equation*}
			where $\epilonfat$ corresponds to a sampled value $\samplestandardnormal{\epilonfat}$ and $\odot$ is element-wise multiplication. The procedure to obtain $\ct$ is analogous to $\zt^m$ which we described above.
			
			
		Because of this probabilistic approach, a single patch of data $\xt$ will have multiple representations $\zt^M$, providing increased variance in the representations. This can potentially benefit downstream tasks, particularly when labelled data in scarce \cite{weiRecentAdvancesVariational2021}, leading to improved performance. % TODO: this benefit should be moved to benefits section.
			

		
		
\section{The learning objective}
	Instead of training the neural network end-to-end with a global loss function, the network is split up into modules, which each are optimised greedily with their own personal loss function. Through the introduction of the novel \textit{Variational-InfoNCE} loss, mutual information between temporally nearby representations is maximised, while regularising the latent space to be approximate to the standard Gaussian $\standardnormal$. The Variational-InfoNCE loss is defined as follows:
	
	\begin{equation} % variational_gim_loss % TODO: die k's is niet echt correct/onvolledig
		% \mathcal{L}(\ztk^{m-1}, \zt^{m-1}) = 
		\Lvnce^m =
		\underbrace{\reconstrgim}_{\text{Maximise } I(\ztk^m, \zt^m)} + \underbrace{\beta ~ \latentspaceconstraintgim}_{\text{Regularisation}}
		\label{eq:variational_gim_loss}
	\end{equation}

	$m \in \naturalset$ refers to the $m$'th module. $k \in \naturalset$ corresponds to the number of patches in the future the similarity score $\fkm$ must rate. $\ztk^m$ and $\zt^m$ are encoded samples produced by $g_{enc}^m(\ztk^{m-1})$ and $g_{enc}^m(\zt^{m-1})$, respectively. $X$ is a set of samples ${ \left\{ \ztk^m, \z_1^m, \z_2^m, \dots \right\} }$ where $\zj^m \neq \ztk^m$ are random samples.


	The similarity score $f_k^m(\cdot)$'s definition is identical to \cite{lowePuttingEndEndtoEnd2020}:
	
	$$ f_k^m(\ztk^m,\zt^m) = \exp({\ztk^m}^TW_k^m\zt^m) $$
	
	$\Lvnce^m$ consists of two terms. The first term ensures that encodings of temporally nearby patches contained maximised mutual information. The second ensures that those encodings are all close to the standard normal $\standardnormal$. Finally, $\beta$ is a hyper-parameter which decides the relative importance between the two terms. $\beta >> 1$ will weight more importance to regularisation, but may result in posterior collapse \cite{lucasUnderstandingPosteriorCollapse2022}. On the other hand $\beta \approx 0$ will attach more importance to the mutual information maximisation term while forgetting about the regularisation term. When $\beta = 0$, V-GIM is identical to GIM but with an altered neural network architecture which supports probabilistic encodings.
	
	\subsection{Gradient \textbf{TODO}}
		The gradient of the first term in $\Lvnce^m$ can be approximated through mini-batches, and optimised directly in PyTorch. With regards to the second term, since $\qfromzmneg$ is a Gaussian, a closed form solution exists \cite{kingmaAutoEncodingVariationalBayes2022}, the term can be differentiated without approximated method.
		
		
		%Where the KL divergence for a single sample $x^{(i)}$ is approximated as follows:
		% https://arxiv.org/pdf/1312.6114.pdf, from example
		\begin{equation}
			\frac{1}{2}\sum_{j=1}^J \left( 1 + \log((\sigma_j^{(i)})^2) - (\mu_j^{(i)})^2 - (\sigma_j^{(i)})^2 \right) 
		\end{equation}
		
		\begin{equation} % REAL BUT MUST REMOVE THE (i)	
			\kl{\normal}{\standardnormal} = \latentspaceconstraintclosedform
		\end{equation}
		
		
		where $z^(i,l) = \sigma ^{(i)} \odot \epsilon^{(l)}$ and $\epsilon^(l) \mathcal{N}(0, I)$
		
		\textbf{todo: variables should maybe be bold.}
		
		
		%	TODO: DUS DIE KL DIVERGENCE CLOSED FORM NOTATIE, MAAR BIJ SAMPLING ZOU OOK DEFINITIE VAN Z = MU + SIGMA*ERR GEVEN
	
	
	\subsection{Continuous space around the origin}
		% around N()
			As we discussed earlier, the representations or encodings $\sample{\zt^m}~ \qfromzmneg$ generated by each module $m$ are samples from a Gaussian distribution (which may not necessarily be the standard normal). These samples are optimised to be as close as possible to the standard normal $\standardnormal$. 
	
		% smooth changes
			Consider $\ztk^{m-1}$ and $\zt^{m-1}$ which each serve as input for a fully trained module $\gencm$. These two inputs are temporally nearby, and thus, due to the slowly varying features assumption \cite{zhangSlowFeatureAnalysis2012} have a lot of information in common. This means that the correspondence score of their encodings, estimated by the scoring function $\fkmblank$, should also be high. However, as depicted in \ref{fig:gaussian-neighbourhood}, the encodings for $\ztk^{m-1}$ correspond an entire space $ \{ {\ztk^m}^{'},~{\ztk^m}^{''},~\dots \}$ centred around a particular mean vector $\mufat$. If $\Lvnce^m$ is optimal, this means that given $\zt^m$, $f_k^m({\ztk^m}^{'}, {\zt^m})$ should be  large, but also $f_k^m({\ztk^m}^{''}, {\zt^m})$, while remaining small for random encodings $\zi^m \neq \ztk^m$. The correspondence scores must thus be similar for all encodings in a particular neighbourhood, meaning they all have similar mutual information to $\zt^{m}$. This is important, because it will ensure smooth transitions in the latent space. Furthermore, optimising this loss function also maximises the mutual information between outputs of successive modules $I(\zt^{m-1}, \zt^{m})$ \cite{lowePuttingEndEndtoEnd2020}, and the smooth transitions will reflect on the original representations $\xt$.			 
		
		\begin{figure} % gauss neighbourhood
			\centering
			\includegraphics[width=0.7\linewidth]{"gaussian neighbourhood"}
			\caption{}
			\label{fig:gaussian-neighbourhood}
		\end{figure}
		
		% gaps
			Finally, since the set of encodings $ \{ {\zt^1}^{'},~{\zt^1}^{''},~\dots \}$ from a single patch $\xt$ corresponds a large neighbourhood in the latent space, and since the latent space is fairly small (standard normal) representation distributions from different data points are likely to be pushed around, trying to utilise the limited space as best as they can. This results in a less likely chance of obtaining holes in the latent space.
			
			The end result is a continuous space around the origin, which is a crucial observation. It will serve as the main argument for why V-GIM's representations are interpretable, while traditional techniques such as CPC and GIM do not have these guarantees. %Additionally, if a decoder is trained on this space and a latent representation is chosen that the decoder has never seen before, the smooth transitions will ensure that the decoder will generalize well to unseen data.
			


		
	
	
	
\section{Computational benefits}
Each module in V-GIM is trained through the variational InfoNCE loss. As a result, latent space constraints are posed on the outputs from the final module, but also in between modules. Hence, many of the benefits we discuss are applicable to both the output and intermediate layers of the ANN architecture.


\subsection{Representations}
	\subsubsection{Interpretability of final and intermediate representations}
		
		
		
		
		While VAE's allow for interpretability by training a decoder, V-GIM does not require one for training, simplifying its architecture. However, a decoder can still be appended to each module to observe the content stored in each representation. 
		
		
		
		why only possble if standard gaussian.
			else decoder may receive inputs dissimilar from what it has seen before and it may not generalise well to the interpolated values.
	
	
	Interpretability of latent representations:
		Encodings are samples from a distribution, which is optimised to be approximate to standard normal.
		- Entire latent space will be close to center, which prevents "gaps" in the latent space.

		- Interpolation --> adds decoder
		
		- ALSO NO EXPLICIT NEED FOR A DECODER. THIS REMAINS, SO ARCH CAN BE SIMPLER.
	
	Better generalisation for downstream tasks
		- Overfitting: reduction of required labelled data needed. Similar data is similar region, the kl divergence makes regions bigger.
		% copied from a bit higher
		% Because of this probabilistic approach, a single patch of data $\xt$ will have multiple representations $\zt^M$, providing increased variance in the representations. This can potentially benefit downstream tasks, particularly when labelled data in scarce \cite{weiRecentAdvancesVariational2021}, leading to improved performance. % TODO: this benefit should be moved to benefits section.

		Overfitting during inference:
		- The same datapoint has multiple (similar) representations, such that learning techniques for downstream tasks will not be able to "memorise" the latent space as easily.
		
		- Holes: more predictable inference, such that unseen data is more likely to be near clusters. And thus downstream tasks receive latents that are more similar to what is seen before.
		= better generalisation
		


	Batch normalisation: is useful for following modules, BUT ALSO DOWNSTREAM TASKS!
		- built in batch normalisation mechanism
		- During training similar behaviour to batch normalization in-between layers
	
	- Independent latent dimensions
		
	- GIM advantages remain: maintain the benefits such as smaller networks that can learn indep, 
	
	
	
	

	







%\textbf{Other sources:} \\
%!!! Abstract on VAE: The fundamental idea in VAEs is to learn the
%distribution of data in such a way that new meaningful data with more intra-class variations can be generated
%from the encoded distribution.
%The ability of VAEs to synthesize new data with more representation variance
%at state-of-art levels provides hope that the chronic scarcity of labeled data in the biomedical field can be
%resolved.
%--> and thus for downstream tasks, has a way of obtaining more labelled data? --> better generalisation


%The goal of representation learning is to be useful for downstream tasks. The most important meta-prior is called ‘disentanglement’ which is an unsupervised learning technique that breaks down, or disentangles, each feature into narrowly defined variables and encodes them as separate dimensions 

%Intuitively, a factorial code disentangles the individual elements that were originally mixed in the sample, just as
%humans recognize complex things by disentangling independent elements. If the dimensions of the latent vector are
%independent of each other, it is factorial disentangled, i.e., a
%good representation. VAEs have made such nonlinear latent
%variable models tractable for modeling complex distributions,
%and efficient extraction of relevant biological information
%from learned features for biological data sets, referred to as
%unsupervised representation learning
%https://ieeexplore.ieee.org/stamp/stamp.jsp?tp=&arnumber=9311619








%We show that the Beta-VAE outperforms principal component analysis (PCA) and learns interpretable and independent representations of the generative factors of variance in the spectra %https://pubs.acs.org/doi/pdf/10.1021/acs.jpclett.2c01328
%



\chapter{Experiments}

\section{Greedy Infomax model}
\begin{itemize}
	\item train CNN
	\item architecture: 4 layers ...
	\item dataset
\end{itemize}


\subsection{Dataset}
The Greedy Infomax model is trained on speech data. The model takes as input a raw speech signal of a fixed length and outputs a latent representation for that signal. The dataset is split up into 729 training files and 122 test files. In each file consists of a single spoken sound consisting of three consonants and three vowels, where the consonants and vowels alternate each other. Some Examples are the sounds "gi-ga-bu" and "ba-bi-gu". All the sounds are spoken by the same person, at a constant and peaceful \textbf{todo: describe emotional aspects of speech audio}. 

The following transformations are applied to the audio files. Although the original contains a sample rate of 441 Khz, the audio files are downsampled to 16 Khz, matching the sample rate used by Löwe \cite{lowePuttingEndEndtoEnd2020}. This significantly reduces the size of the latent representations, and thus the required amount of VRAM during training. Additionally, two types of noise are added to the data. We apply Gaussian white noise, at different decibels ranging between zero and fifteen \textbf{TODO ...}.

- also background noise from dataset. is a way to enlargen our dataset. 

%class RandomBackgroundNoise:
%def __init__(self, target_sample_rate, noise_dir, min_snr_db=0, max_snr_db=15):
%self.target_sr = target_sample_rate
%self.min_snr_db = min_snr_db
%self.max_snr_db = max_snr_db
%
%if not os.path.exists(noise_dir):
%raise IOError(f'Noise directory `{noise_dir}` does not exist')
%# find all WAV files including in sub-folders:
%self.noise_files_list = list(pathlib.Path(noise_dir).glob('**/*.wav'))
%if len(self.noise_files_list) == 0:
%raise IOError(
%f'No .wav file found in the noise directory `{noise_dir}`')

- Each audio file is cut to have length \textbf{10240}. Additionally, 


% split audio files into three separate files (one for each syllable)
Since the recordings are very consistent in loudness and are noise free, we can split up the files per syllable, obtaining three files per original sound (one for each syllable).

A sliding window is used of size 0.02 seconds. With a sample rate of 22050, this corresponds to roughly 500 samples per window. The maximum is computed for each window. Speech signals can then be split up when a severe dip happens in the signal. Regions where the amplitude is greater than 0.2 are considered \textbf{klinkers}, the regions with with lower values are considered \textbf{medeklinkers}. Apart from a few edge cases, this technique worked well enough for this purpose. In those cases, the splitting points closest to the one-third and two-third splitting points were considered.

\textbf{ERR: OUDE AFBEELDINGEN ZIJN WEG}
%\begin{figure}[h]
%	\centering
%	\includegraphics[width=0.7\linewidth]{"../../../../../../../../../GitHub/thesis-fabian-denoodt/GIM/datasets/gigabo/split up graphs/test/bababa_1"}
%	\caption{}
%	\label{fig:bababa1}
%\end{figure}
%
%\begin{figure}[h]
%	\centering
%	\includegraphics[width=0.7\linewidth]{"../../../../../../../../../GitHub/thesis-fabian-denoodt/GIM/datasets/gigabo/split up graphs/test/bababa_1_split"}
%	\caption{}
%	\label{fig:bababa1split}
%\end{figure}



note: we do need a hard threshold which is based on the signal's intentisity level. One could consider the alternative approach of looking at the gradient at each point and selecting the points with largest negative gradient. This will work in many cases, however, not for temporal envelops which gradually move towards zero, s.t: \ref{fig:example where gardient doesnt work}. Instead we use a dynamic threshold. This threshold is computed by creating transforming the signal into bins of 90'th percentile, creating a histogram of the single signal and applying otsu's image segmentation algorithm to obtain the threshold of that single audio sample.
We also tried directly applying otsu to the moving average and maximum of the bins. This either gave a threshold that was too small or too large. the 90th percent resulted in an acceptable compromise.


\begin{figure}[h]
	\centering
	\includegraphics[width=0.7\linewidth]{screenshot011}
	\caption{}
	\label{fig:example where gardient doesnt work}
\end{figure}



Example where the explained strategy does not work: %\ref{fig:badadi1}
%\begin{figure}[h]
%	\centering
%	\includegraphics[width=0.7\linewidth]{"../../../../../../../../../GitHub/thesis-fabian-denoodt/GIM/datasets/gigabo/split up graphs/train/badadi_1"}
%	\caption{}
%	\label{fig:badadi1}
%\end{figure}/


Reference images for in the text:
\begin{figure}
	\centering
	\includegraphics[width=0.7\linewidth]{screenshot012}
	\caption{}
	\label{fig:max sliding window}
\end{figure}

\begin{figure}
	\centering
	\includegraphics[width=0.7\linewidth]{screenshot013}
	\caption{}
	\label{fig:90th percentile}
\end{figure}

\begin{figure}
	\centering
	\includegraphics[width=0.7\linewidth]{screenshot014}
	\caption{}
	\label{fig:mask}
\end{figure}

\begin{figure}
	\centering
	\includegraphics[width=0.7\linewidth]{screenshot017}
	\caption{}
	\label{fig:full sound wave adjusted yaxis}
\end{figure}


\begin{figure}
	\centering
	\includegraphics[width=0.7\linewidth]{screenshot016}
	\caption{}
	\label{fig:split up sound wave}
\end{figure}


\begin{figure}
	\centering
	\includegraphics[width=0.4\linewidth]{screenshot018}
	\caption{}
	\label{fig:histogram}
\end{figure}


audio padded to maximum length. (added zeros in front and back)




\subsection{Training and validation during training}
Although loss is used as evaluation metric, it only shows part of the picture. The main objective for the representations is to obtain some form of "decoupled/dis-entanged" features that are more easily separable. This evaluation is done by projecting the latent representations to a 2D plane, via t-SNE.
Then datapoints are coloured in depending on their the syllable that was pronounced, eg: "gi" or "ga".

Training is done speech signals of fixed length, eg 8,800 samples. Notice however, that the neural network only makes use of convolutional neural networks layers and GRU's, no fully connected layers. The input dimensions can therefore be variable during inference. Only the number of channels in the latent representations should be constant, but the length can change.

During inference, (in this context obtaining the latent representations for our input signals), depending on the length of the input signal, the length of the output latent representation will differ.
If we wish to look at how separable latent representations are for syllables, the length can be variable. Some input sounds could be 6,600 samples, while others 8,800 samples. We therefore pad the syllables with zeroes in front and end of the signal, to obtain fixed length of equal to that of the longest syllable; 8,800 samples.

Training happens on longer data samples, and every \textbf{X} epochs t-SNE visualisations are made to observe evolutional of dis-entanglement.




\section{Decoders for Greedy Infomax}
Objective: understand GIM learned representations. To do so we first discuss the model's architecture. To verify whether GIM does in fact contain the necessary information, we reconstruct the original speech signal from a GIM representations. We define train a decoder model for each of the four layers of the GIM model.

\begin{itemize}
	\item Layer 1 ... 4
	\item Dataset from Bart
	\item Reduce audio to 16khz, reducing the size of the dataset and speeding up training time
	\item Add background noise to the data.
	\item Use this dataset: openslr.org instead of white noise as is more representative.
\end{itemize}






\section{Feature selection for visualisation}
The GIM architecture consists of 4 layers, each layer 512 kernels
Due to the large number of learned kernels, we filter based on standard deviation. This will result in looking at weights which are sensitive to particular features.


\section{Variational contrastive predictive coding}
\begin{figure}[h]
	\centering
	\includegraphics[width=0.7\linewidth]{architecture}
	\caption{}
	\label{fig:architecture}
\end{figure}
Figure \ref{fig:architecture} displays how via contrastive predictive coding an input speech signal is transformed into two latent vectors. The two vectors combined describe a Gaussian distribution for each feature of the latent representation. One vector corresponds to the means and the other to the standard deviations of the latent distributions. We variational autoencoders assume independent latent features, such that the covariance matrix is non negative on the diagonal and zero off the diagonal. This allows the covariance matrix to be discribed using a single vector. We again, make use of this (plausible incorrect?) assumption of having independent latent features.


\section{CHATGPT ANSWERS FOR VAE}
Yes, you can incorporate the regularization component from a VAE into the contrastive predictive coding (CPC) loss function to encourage the learned latent representations to be Gaussian distributed. This can be achieved by adding a KL divergence term between the learned latent distribution and a Gaussian prior distribution to the CPC loss function.

The modified loss function would be:

\begin{equation}
	\mathcal{L}{N+KL} = - \sum{k} \mathbb{E}{\textsubscript{X}} \left[ log \frac{f_k(z{t+k}, c_t)}{\sum_{z_j \in X} f_k(z_j, c_t)} \right] + \lambda \text{KL}(q(z|X) || \mathcal{N}(0, I))
\end{equation}

where $q(z|X)$ is the learned distribution of latent representations, and $\mathcal{N}(0, I)$ is the standard Gaussian prior distribution. The hyperparameter $\lambda$ controls the weight of the KL divergence term in the overall loss function.

To compute the KL divergence term, we need to calculate the mean and covariance of the learned latent distribution $q(z|X)$. Let $\mu$ and $\Sigma$ be the mean and covariance of $q(z|X)$, respectively. Then the KL divergence can be computed as:

\begin{equation}
	\text{KL}(q(z|X) || \mathcal{N}(0, I)) = \frac{1}{2} \left( \text{tr}(\Sigma) + \mu^T \mu - k - \log \det(\Sigma) \right)
\end{equation}

where $k$ is the dimensionality of the latent space.

The function $f_k(z_j, c_t)$ can be defined using a linear layer as follows:

\textbf{code}
%W_k = torch.nn.Linear(hidden_size, latent_size, bias=False)
%f_k = lambda z, c: torch.exp(torch.matmul(W_k(z), c.T))

Here, `hidden size` is the size of the hidden layer, `latent size` is the size of the latent space, and c is the context vector for a given time step. The function $W_k(z)$ applies a linear transformation to the input z, resulting in a latent vector of size `latent size`.

To incorporate the KL divergence term into the CPC loss function, you can compute the mean and covariance of the learned latent distribution using the torch.mean and torch.var functions, respectively. The modified loss function can then be implemented as follows:

\textbf{code}
%def cpc_loss_with_regularization(z, c, lambd):
%# Compute the CPC loss
%logits = compute_logits(z, c)
%cpc_loss = -torch.mean(torch.log_softmax(logits, dim=-1)[:, 0])
%
%# Compute the mean and covariance of the learned latent distribution
%mu = torch.mean(z, dim=0)
%sigma = torch.var(z, dim=0)
%
%# Compute the KL divergence loss
%kl_loss = 0.5 * (torch.sum(sigma) + torch.matmul(mu, mu.T) - mu.shape[0] - torch.log(torch.prod(sigma)))
%
%# Compute the total loss
%loss = cpc_loss + lambd * kl_loss
%
%return loss

Here, z is the learned latent representation, c is the context vector, and lambd is the weight of the KL divergence term. The `compute logits` function computes the logits for the CPC loss as described in the original CPC paper.

Note that this implementation assumes that the ....

---


To incorporate the regularization component from VAE into the CPC loss function, we can add the Kullback-Leibler (KL) divergence term to the loss function, which will encourage the latent representations to be Gaussian distributed. The KL divergence term measures the difference between the learned latent distribution and the target Gaussian distribution.

The KL divergence term can be defined as:

$$\mathcal{L}{KL} = \frac{1}{2}\sum{i=1}^d (\mu_i^2 + \sigma_i^2 - \log(\sigma_i^2) - 1)$$

where $\mu_i$ and $\sigma_i$ are the mean and standard deviation of the $i$-th dimension of the learned latent distribution, and $d$ is the dimension of the latent space.

To combine the CPC loss and the KL divergence term, we can simply add them together, with a weighting factor $\beta$ to balance their contributions:

$$\mathcal{L} = \mathcal{L}N + \beta\mathcal{L}{KL}$$

where $\mathcal{L}_N$ is the CPC loss function defined in equation (\ref{eq:NCE_loss}), and $\beta$ is a hyperparameter that controls the strength of the regularization term.

The final loss function will encourage the learned latent representations to be both informative for the CPC task and Gaussian distributed, which can be useful for downstream tasks such as clustering or classification.



---

Sure! In the context of contrastive predictive coding (CPC), the `compute logits` function is used to compute the similarity scores (logits) between the predicted future and the actual future.

In more detail, `compute logits` takes in two inputs, z and c, which represent the latent representation and the context for a given time step, respectively. The function computes the similarity score between the predicted future representations and the actual future representations by performing a dot product between the predicted future representation and a set of parameters (i.e., weights) that are learned during training. Specifically, `compute logits` applies a linear transformation to the context c and uses it to score each element in the predicted future z.

The output of `compute logits` is a vector of logits that corresponds to the similarity scores between each predicted future representation and the actual future representation. These logits are then used to compute the contrastive loss, which is a measure of the similarity between the predicted future and the actual future.

The `compute logits` function is a crucial component of the CPC algorithm, as it determines how well the predicted future representations match the actual future representations, which in turn affects the quality of the learned latent representation.




\section{Results}
% Enkele loss curces etc
\subsection{Results CPC Simple v2 w/ 2 modules}

\begin{figure}[h]
	\centering
	\includegraphics[width=0.7\linewidth]{screenshot023}
	\caption{}
	\label{fig:tsne_two_module_kld_0}
\end{figure}


\begin{figure}[h]
	\centering
	\includegraphics[width=0.7\linewidth]{screenshot024}
	\caption{}
	\label{fig:tsne_two_module_kld_0033}
\end{figure}

\ref{fig:tsne_two_module_kld_0}: can see that second model harms performance. We believe this can be explained via the learning rate. \ref{fig:tsne_two_module_kld_0033}, there we see that second module performs better separation, indicating that the intermediate KL convergence constraint (causing the normal Gaussian distributions) also serves as a batch normalisation term, and thus resulting in faster convergence.




\section{GIM: Activations visualisations}

%these were notes from the decoder from file: eval_autoencoder.py
thought for later: its actually weird i was able to play enc as audio as enc is 512 x something
so huh? that means that a lot of info is already in first channel? what do other 511 channels then contain?
"""
Observations:
First layer decoded still contains the same sound, but with some added noise (could be because decoder hasn't trained very).
However, the encoded first layer, still contains the exact sound as the original sound. It is however downsampled a lot -> from 16khz to ~3khz
"""
thought for later: its actually weird i was able to play enc as audio as enc is 512 x something
so huh? that means that a lot of info is already in first channel? what do other 511 channels then contain?



\begin{figure}[h]
	\centering
	\includegraphics[width=0.7\linewidth]{screenshot007}
	\caption{"BA-BA-BA" time domain}
	\label{fig:screenshot007}
\end{figure}

\begin{figure}[h]
	\centering
	\includegraphics[width=0.7\linewidth]{screenshot006}
	\caption{Activations of the sound "BA-BA-BA" through GIM}
	\label{fig:gim latent activations}
\end{figure}

No batch normalisation, so although channels appear to have larger activations than other channels, size of activation does not really say anything about information. eg activations 0.01 could still contain more information than 3.0 activation.

Since the activations from convolutional neural networks, the order is still maintained. Hence, can align activations with original signal.

Observations in latent representations:

\textit{Layer 1:}
The activations of the first decoder still contain a lot of similarity with the original signal, in terms of structure. There is a lot of redundant data within the representation. Eg: the one channel could be replied 

Layer 2

Layer 3:

Layer 4:
Still notices multiple channels which have high activations when signal is has high amplitudes and small activations when amplitude is low. 

Also activations which are high when volume is low. --> indicates that certain kernel weights are sensitive for \textbf{"klinkers"} and other kernels for \textbf{medeklinkers}. see \ref{fig:screenshot008}.

\begin{figure}[h]
	\centering
	\includegraphics[width=0.7\linewidth]{screenshot008}
	\caption{zoomed in}
	\label{fig:screenshot008}
\end{figure}


Observe that activations happen in clusters/sequences. So it is usually a patch of signal samples that cause high activations. This could for instance indicate that both kernels are sensitive for the \textbf{medeklinker} "b", but sensitive for different features. eg the letter B has spoken sound "buh". so maybe one is sensitive for "b" and other for "uh".

Figure \ref{fig:layer4 zoomed in} also nicely shows how different channels have clusters of activations at slightly different times. 

\begin{figure}[h]
	\centering
	\includegraphics[width=0.7\linewidth]{screenshot010}
	\caption{Zoomed in}
	\label{fig:layer4 zoomed in}
\end{figure}



---------------------------------------------

\textbf{T-sne visualisations:}
Multiple models trained, trained GIM, but single module (exact CPC architecture), one with autoregressor and once without autoregressor layer, so only CNN layers. 200 epochs, trained on split up data samples.

1) Only CNN: We observe better linear separability
\begin{figure}[h]
	\centering
	\includegraphics[width=0.7\linewidth]{"cpc architecture ONLY CNN t-SNE_latent_space_GIM"}
	\caption{}
	\label{fig:cpc-architecture-only-cnn-t-snelatentspacegim}
\end{figure}

2) CNN + 1 autoregressor layer:
\begin{figure}[h]
	\centering
	\includegraphics[width=0.7\linewidth]{"cpc architecture CNN + GRU t-SNE_latent_space_GIM"}
	\caption{}
	\label{fig:cpc-architecture-cnn--gru-t-snelatentspacegim}
\end{figure}

3) This is the pure data visualised (no latent representation).
Were at least doing a bit better than the original data, so that's good! our work is not for nothing.
\begin{figure}[h]
	\centering
	\includegraphics[width=0.7\linewidth]{"_ t-SNE_latent_space_Original data"}
	\caption{}
	\label{fig:-t-snelatentspaceoriginal-data}
\end{figure}


4) old GIM with all modules each one layer. l1 .. 5 - cnn, l6 = gru. img shows l5 = cnn:
model can more easily distinguish A's from other \textbf{klinkers}. Partly, it kinda makes sense for GIM to learn to separate \textbf{klinkers}. Since they last longer (longer duration), the loss function will more likely randomly sample a subwindow from the "aa's" than from the \textbf{medeklinker} part.


\begin{figure}[h]
	\centering
	\includegraphics[width=0.7\linewidth]{"_ t-SNE_latent_space_GIM multi module cnn layer 5"}
	\caption{}
	\label{fig:-t-snelatentspacegim-multi-module-cnn-layer-5}
\end{figure}
% 	







\section{Decoder: predictions on test set}
Fig \ref{fig:bagidi1-model29-true-vs-predicted} displays the reconstructed signal from the vocal sound "ba-gi-di". The two images on the left displays the original signal, while the right two images contain the reconstructed signal.  The upper images displays the signals in time domain, the bottom images spectral domain. The reconstructed signal is an audio sample, for instance which is encoded via Greedy Infomax (up to the fourth (and final) convolution layer), this output is then given to a decoder to reconstruct the original signal.

%TODO: BROKE 
%\begin{figure}[h]
%	\centering
%	\includegraphics[width=0.7\linewidth]{"../../../../../../../../../GitHub/thesis-fabian-denoodt/GIM/logs/GIM_DECODER_experiment/MSE + scFFT Loss FFT=10240 Lambda=1.0000000/lr_0.0010000/GIM_L4/predictions_model=29/test/bagidi_1, model=29, True vs Predicted"}
%	\caption{Top left: original, time domain. Bottom left: original }
%	\label{fig:bagidi1-model29-true-vs-predicted}
%\end{figure}




%*******************************************************************************
\chapter{Related work}
\section{Representation learning: explainable}
\section{Variational learning}
There already were a few papers with variational contrastive predictive coding

\section{Links I should investigate}

- Adversarial and Contrastive Variational Autoencoder for Sequential Recommendation
https://arxiv.org/pdf/2103.10693.pdf

- S3VAE: Self-Supervised Sequential VAE for Representation Disentanglement and Data Generation
https://arxiv.org/abs/2005.11437

- Implementation of Sequential VAE
https://github.com/ermongroup/Sequential-Variational-Autoencoder

- Contrastively Disentangled Sequential Variational Autoencoder
https://proceedings.neurips.cc/paper/2021/file/53c5b2affa12eed84dfec9bfd83550b1-Paper.pdf

- Sequential Variational Autoencoders for Collaborative Filtering
https://arxiv.org/pdf/1811.09975.pdf

- !! Variational noise contrastive estimation:
https://arxiv.org/abs/1810.08010

\chapter{Discussion}

-- 
decaying learing rate:
	we train using decaying lr, because models must first learn distributions and goes too slow if lr is too small.
	and a learning rate scheduler
	ExponentialLR
	decay rate 0.995

---
batch norm:
 - sindy didn't have issues of batch norm, but believe this is because each module consisted of a single layer, ours contain a number of layers. potentially: outputs from first module change too fast for second module to catch up.




while GIM argues to resolve memory constraints, not entirely true. In fact we even countered the opposite as containing multiple neural networks, each with their own personal loss function (the loss function is based on fk which contains parameters that must be learned), and thus for early layers where the sequence is still long, a lot of memory is required. We went for a compromise on GIM by splitting up the architecture in merely two modules, significantly reducing the memory constraints.


---
The second module in GIM clearly doesn't have as much effect. This can be explained because there may not be as much common information anymore between the patches. There may be a source that says that cpc learns low level features, but the second module is supposed to learn more high level features, which cpc may have trouble with?
---

Future work:
 - Related work in VAE shows that gradually increasing regularisation term, results in better disentengledment, while avoiding posterior collapse. could have a kldweight scheduler.
 
- not constrained solely to InfoNCE loss, the GIM architecture could work for other losses too that allow for greedy optimisation.


- I didn't add an autoregressor as i didn't find a performance benefit. Potentially, with larger architecture could further improve performance.



----
Towards production setting:
	encodings are thus optimised to be close the standard normal. When in a production environment and new data is given, could in fact have an idea of how well generalisation to the production data: eg via anomaly detection if encodings are too far away from center. 
	= gives automated way of verifying generalisation.
	
	can then maybe see to which data that doesn't generalise well via outliers.
	
----


future work:
- disentanglement should do more investations


---
GIM: Modular training
could incrementally increase numb of modules and observe performance increase for downstream tasks.
based on this, could find smallest gim architecture depth which satisfies required accuracies.

----
interpretability:
most dimensions sensitive around 75 to 150 hz. this is as expected as the adult man speaks around 80 to 180 hz.

---
interpretability is only as good as the decoder. if a shitty decoder doesn't construct well, doesn't necessarily give correct conclusions about V-GIM.


---
Future work:alternatieve prior (zie related work.)






\begin{itemize}
	\item Explainability of latents is dependent on the performance of the decoder.
	\item Intermediate loss function with kld resulted in similar behaviour as batch normalisation. Resulting in faster convergence than without kld.
	\item We observed no quality loss in the learned representations. Data was equally easily separable.
\end{itemize}



\bibliographystyle{apalike}
\bibliography{references}


\begin{appendices}
	\chapter{Syllable classification through histogram segmentation} \label{appendix:split_syllables}
	The syllable classifiers in sections \ref{cha:exper_classifier} and \ref{cha:generalisation_study} take as input a fixed-length speech wave of a syllable. However, the words in the speech dataset are not yet split up into syllables, and we must do this ourselves.
	
	To split up the sound files in the dataset, we make use of Otsu's histogram segmentation algorithm \citep{otsuTlresholdSelectionMethod1979}. Afterwards, the split speech waves are padded with zeros in the front and back such that each wave has the same length. The files are padded to contain 8800 samples (at a sample rate of 16 kHz.).
	
	Otsu's algorithm is traditionally used for segmentation applications on greyscale images. Given a single image, it chooses a particular intensity threshold and classifies all pixels into one of two classes depending on whether their intensity values are smaller or larger than the threshold. The threshold is chosen to be the one that minimises the intra-class variance \citep{binduEfficientMedicalImage2012}. 
	
	We explore an alternative domain for the algorithm and use it to find an amplitude threshold in the speech waves. The recordings are consistent in loudness and each file contains exactly three syllables of the form ``consonant - vowel". It, therefore, suffices to find a single threshold per audio file that classifies the amplitudes as vowel or consonant.
	
	We first max pool the audio waves with a window size of 0.02 seconds. This emphasises the discrepancy in amplitude between vowels and consonants, as shown in figure \ref{fig:max sliding window}. The threshold that minimises the variance for the two classes is computed using a histogram, shown in figure \ref{fig:histogram}, in this case, 0.10. 
	
	The time windows with amplitude smaller than 0.10 are classified as consonants and the other time windows as vowels. However, when directly using the threshold on the max pooled speech wave we observed that not all consonants were detected, and thus too much speech wave was being classified as consonant. Instead, we found that applying the threshold (still obtained from the max pooled speech wave) to the 90th percentile was a good compromise between classifying speech chunks as vowels or consonants. Once the vowels and consonants are obtained, syllables are computed at transition points going from vowel to consonant, shown in figure \ref{fig:full sound wave adjusted yaxis}. The transition points mark the ending of a syllable.

	Apart from a few edge cases, this technique worked well enough for our purposes. In the cases where more than three vowels were obtained, the transition points closest to the one-third and two-third duration mark were considered instead.

		
		\begin{figure}
			\centering	
			\begin{minipage}{\linewidth}
				\centering
				
				\begin{subfigure}{\linewidth}
					\centering
					\includegraphics[width=0.5\linewidth]{screenshot017}
					\caption{Sound wave for ``ba ba bi", which should be split up into ``ba", ``ba and ``bi".}
					\label{fig:full sound wave adjusted yaxis}
				\end{subfigure}
				
				\vspace{1em}
				
				
				\begin{subfigure}{\linewidth}
					\centering
					\includegraphics[width=0.5\linewidth]{screenshot012}
					\caption{Resulting speech wave after max pooling. The threshold will be computed using this speech wave.}
					\label{fig:max sliding window}
				\end{subfigure}
				
				\vspace{1em}
				
				\begin{subfigure}{\linewidth}
					\centering
					\includegraphics[width=0.3\linewidth]{screenshot018}
					\caption{Threshold obtained using Otsu's algorithm. The red dashed line corresponds to the threshold.}
					\label{fig:histogram}
				\end{subfigure}
				
				\vspace{1em}
				
				\begin{subfigure}{\linewidth}
					\centering
					\includegraphics[width=0.5\linewidth]{screenshot013}
					\caption{90'th percentile speech wave (computed using moving window of 0.02 seconds). The threshold is applied to this speech wave. Amplitudes larger than the threshold are considered vowels and smaller values are consonants.}
					\label{fig:90th percentile}
				\end{subfigure}
			
				\vspace{1em}				
				
				\begin{subfigure}{\linewidth}
					\centering
					\includegraphics[width=0.5\linewidth]{screenshot014}
					\caption{Obtained mask from applying the threshold. The x coordinates going from one to zero are transition points and mark the end of a syllable. There are three potential points in this image. Since each audio file contains three syllables (and thus a maximum of two transition points), the two \textit{true} points are selected based on their distance from the one-third and two-third x coordinate. In this case, the last transition point will be discarded.}
					\label{fig:mask}
				\end{subfigure}
				
				\vspace{1em}
				
				\begin{subfigure}{\linewidth}
					\centering
					\includegraphics[width=0.5\linewidth]{screenshot016}
					\caption{The three obtained syllables after cutting the speech wave at the two transition points.}
					\label{fig:split up sound wave}
				\end{subfigure}
			\end{minipage}			
		\end{figure}
	



	




%	\section{GIM: Activation visualisations}
%	
%	%these were notes from the decoder from file: eval_autoencoder.py
%	thought for later: its actually weird i was able to play enc as audio as enc is 512 x something
%	so huh? that means that a lot of info is already in first channel? what do other 511 channels then contain?
%	"""
%	Observations:
%	First layer decoded still contains the same sound, but with some added noise (could be because decoder hasn't trained very).
%	However, the encoded first layer, still contains the exact sound as the original sound. It is however downsampled a lot -> from 16khz to ~3khz
%	"""
%	thought for later: its actually weird i was able to play enc as audio as enc is 512 x something
%	so huh? that means that a lot of info is already in first channel? what do other 511 channels then contain?
%	
%	
%	
%	\begin{figure}[h]
%		\centering
%		\includegraphics[width=0.7\linewidth]{screenshot007}
%		\caption{"BA-BA-BA" time domain}
%		\label{fig:screenshot007}
%	\end{figure}
%	
%	\begin{figure}[h]
%		\centering
%		\includegraphics[width=0.7\linewidth]{screenshot006}
%		\caption{Activations of the sound "BA-BA-BA" through GIM}
%		\label{fig:gim latent activations}
%	\end{figure}
%	
%	No batch normalisation, so although channels appear to have larger activations than other channels, size of activation does not really say anything about information. eg activations 0.01 could still contain more information than 3.0 activation.
%	
%	Since the activations from convolutional neural networks, the order is still maintained. Hence, can align activations with original signal.
%	
%	Observations in latent representations:
%	
%	\textit{Layer 1:}
%	The activations of the first decoder still contain a lot of similarity with the original signal, in terms of structure. There is a lot of redundant data within the representation. Eg: the one channel could be replied 
%	
%	Layer 2
%	
%	Layer 3:
%	
%	Layer 4:
%	Still notices multiple channels which have high activations when signal is has high amplitudes and small activations when amplitude is low. 
%	
%	Also activations which are high when volume is low. --> indicates that certain kernel weights are sensitive for \textbf{"klinkers"} and other kernels for \textbf{medeklinkers}. see \ref{fig:screenshot008}.
%	
%	\begin{figure}[h]
%		\centering
%		\includegraphics[width=0.7\linewidth]{screenshot008}
%		\caption{zoomed in}
%		\label{fig:screenshot008}
%	\end{figure}
%	
%	
%	Observe that activations happen in clusters/sequences. So it is usually a patch of signal samples that cause high activations. This could for instance indicate that both kernels are sensitive for the \textbf{medeklinker} "b", but sensitive for different features. eg the letter B has spoken sound "buh". so maybe one is sensitive for "b" and other for "uh".
%	
%	Figure \ref{fig:layer4 zoomed in} also nicely shows how different channels have clusters of activations at slightly different times. 
%	
%	\begin{figure}[h]
%		\centering
%		\includegraphics[width=0.7\linewidth]{screenshot010}
%		\caption{Zoomed in}
%		\label{fig:layer4 zoomed in}
%	\end{figure}
	
	

\end{appendices}

\end{document}
