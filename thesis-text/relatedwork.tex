
%*******************************************************************************
\chapter{Relation to existing work}

%Contrastive Predictive Coding has seen success in recent years, in addition we have GIM which contributes to this. TODO: MISS ZOU DIT EIG OOK GEWOON IN INTRODUCTIE KUNNEN?
In recent years, CPC has been shown to be a successful self-supervised learning approach in a wide range of domains \cite{stackeEvaluationContrastivePredictive2020, dehaanContrastivePredictiveCoding2021, luSemiSupervisedHistologyClassification2019, bhatiSegmentalContrastivePredictive2021b, deldariTimeSeriesChange2021, henaffDataEfficientImageRecognition2020}. Löwe et al. contribute to this work by showing that modules can be trained greedily, each with their own instance of the InfoNCE loss, enabling modules to be trained in parallel, or sequentially. Moreover, Wang \cite{meihanwangSpeechRepresentationLearning2019} empirically demonstrates that GIM learns good speech representations for the speech dataset we use in our own experiments \ref{cha:experim_details_vgim}.

However, the underlying representations obtained from optimising the InfoNCE loss have not yet been studied in great detail. In this thesis we shed light onto this, through the introduction of a constraint to the latent space, resulting in a space which is easier to analyse and understand.

We have studied the latent representations obtained from maximising the InfoNCE objective. 
We achieved this through the introduction of V-GIM, a self-supervised representation learning approach with the same InfoNCE objective, but with an additional constraint to the latent space resulting in better interpretable representations. Such that a decoder could be trained and predict meaningful ...
	
\section{Explainable AI}
	%While multiple XAI techniques exist, they work in different paradigms, usually attempt to visualise 
	This is a vastly different approach from existing techniques in explainable AI. Bai et al. group the techniques in three categories \cite{baiExplainableDeepLearning2021}; attribution-based methods, non-attribution-based methods and uncertainty quantification.
	1) tries to attribute a prediction to its input features. eg used for images and can highlight regions contribute to the decision.
	
	
	
	X et al. group XAI techniques in x cateogories
		In the field of Explainable AI multiple paradigms exist, ranging from activation heatmaps ...
	These techniques give insights in visual domain, but lack in other domains such as speech domain where heatmaps be harder to gain insights from.
	
		
	
	- 

---
Explainable ANNs:
	- diff paradigms, by looking at heath maps and lr etc
	- our work is in fact new paradigm, by adding constraints to the optimisation metric, resulting in better understandable latent representations
	- Explainable deep learning methods survey: \cite{baiExplainableDeepLearning2021} (attribution and non attribution, zie mijn draft.dox) --> probeert contribution van elke feature te linken. dat zijn technieken die werken voor foto's of feature vectors, maar voor puur sequential audio is moeilijker bruikbaar.
	
	Maybe exists other techniques that change the ANN resulting in better explainable. (eg pruning?)
	- methodology to remove features that do not contribute to accuracy. (feature selection) with interpretability motivations. \cite{glorfeldMethodologySimplificationInterpretation1996}
	
	
Explainable learning in speech data


Summarise: our method: sequential data/speech data, interpretable, representation learning, disentanglement

\section{Mutual information and interpretable Representation learning}

	% eerst MI inleiden
	If we further continue in the Mutual Information Maximisation research line, GIM and Deep InfoMax both obtained representations through mutual information maximisation. GIM maximises the mutual information between temporally nearby patches, assuming common information between nearby data \cite{lowePuttingEndEndtoEnd2020}. It is also the basis for our own contribution, V-GIM. Similarly, Deep InfoMax considers an ANN encoder model which maximises the mutual information between input and output. This is achieved by incorporating knowledge about locality in the representations, resulting in locally-consistent information across structural locations \cite{hjelmLearningDeepRepresentations2019}.
	InfoGAN \cite{chenInfoGANInterpretableRepresentation2016} is an extension to the GANs and also learns representations by maximising the mutual information between the generated data and small subsets of the latent variables in the latent representations. In addition, InfoGAN's learning approach results in disentangled and interpretable representations, where manipulating an individual dimension in the latent space, results in changes to a specific feature of the generated data, while other features remain unchanged. This is achieved through a modification of the traditional minimax loss function in GANs.
	
	Further continuing in representation learning methods which incorporate an interpretability mechanisms, Timeline uses RNNs with an attention mechanism to aggregate sequential health to interpretable representations \cite{baiInterpretableRepresentationLearning2018}. This interpretability is achieved through analysis of weights associated to different medical codes. Similarly, Agrawal and Ganapathy use relevance weighting on raw speech data, allowing for interpretation of the representations during forward propagation \cite{agrawalInterpretableRepresentationLearning2020}.
	
	
	
	% in MI domain? Deep InfoMax [17] and Greedy InfoMax
		

\section{Alternative priors and posteriors in VAEs}	% TODO: MAYBE THIS SHOULD BE MOVED TO DISCUSSION/FUTURE WORK?
Taking inspiration from VAEs, V-GIM aims to minimise the KL-divergence with its posterior distributions $\qphizx$ and prior $\prior=\standardnormal$. Doing so results in a latent space which can be better understood. In recent years multiple contributions were introduced to VAEs in relation to different priors and posteriors.

In VAEs the posterior $\qphizx$ is in fact an approximate for the true posterior $\pphizx$ \cite{odaiboTutorialDerivingStandard2019}. This approximate posterior is most commonly chosen to be a simple factorised Gaussian for easier mathematical derivations, however, this is typically an oversimplification of the true posterior \cite{nalisnickApproximateInferenceDeep}. Consequently, Kingma and Welling show that the approximate posterior can be extended to a Gaussian with full covariance matrix \cite{kingmaIntroductionVariationalAutoencoders2019}. Additionally, Nalisnick et al. propose a Gaussian mixture model (a combination several Gaussians) as approximate posterior, enabling multimodal posterior distributions \cite{nalisnickApproximateInferenceDeep}.

Continuing in the line of Gaussian mixture models, extensions to alternative priors have been explored as well. Lee et al. and experiment with Gaussian mixture model priors \cite{leeMetaGMVAEMixtureGaussian2021, guoVariationalAutoencoderOptimizing2020}, resulting in improved performance. Additionally, Tomczak and Welling introduce VampPrior, which choses the prior to be a mixture of variational posteriors, resulting in improved model performance and reducing issues related to useless dimensions \cite{tomczakVAEVampPrior2018}. This is a known issue in VAEs which we observed V-GIM suffers from as well.
	


%---
%CPC and GreedyInfomax:
%	- Bart's previous thesis student,
%		The same dataset was used by X, Y, for speech learning. X trained a convolutional neural network with the speech recognition task. Y further continued this work by developing a model, based on Greedy Infomax. Showing that learning is feasible. We contribute to Y’s ideas by inspecting the representations that were learned from GIM.
%		--> Showing that greedy learning with the corpus is feasible.





\section{Representation learning: explainable}
\section{Variational learning}
There already were a few papers with variational contrastive predictive coding

\section{Links I should investigate}
- S3VAE: Self-Supervised Sequential VAE for Representation Disentanglement and Data Generation
https://arxiv.org/abs/2005.11437

- Implementation of Sequential VAE
https://github.com/ermongroup/Sequential-Variational-Autoencoder

- Contrastively Disentangled Sequential Variational Autoencoder
https://proceedings.neurips.cc/paper/2021/file/53c5b2affa12eed84dfec9bfd83550b1-Paper.pdf

- Sequential Variational Autoencoders for Collaborative Filtering
https://arxiv.org/pdf/1811.09975.pdf

- !! Variational noise contrastive estimation:
https://arxiv.org/abs/1810.08010