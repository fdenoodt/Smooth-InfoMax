
%*******************************************************************************
\chapter{Relation to existing work}

We have studied the representations obtained by maximising the InfoNCE loss function. We achieved this through the addition of a regularisation term to the loss function, resulting in a space that is easier to analyse and understand. In this chapter, we give an overview of the existing literature that is relevant to our research. We begin with a discussion of representation learning techniques related to mutual information maximisation, and slowly digress into interpretable representations. Furthermore, we give an overview of the existing regularisation techniques used for both improved generalisation and better representations. Finally, we give an overview of alternative priors that have been introduced to the VAE framework in the past decade.


\section{Mutual information and interpretable Representation learning}
	V-GIM is based on the ideas of mutual information maximisation introduced in CPC and GIM \cite{lowePuttingEndEndtoEnd2020, oordRepresentationLearningContrastive2019}. This is achieved by maximising the mutual information between temporally nearby patches, assuming common information between nearby data \cite{lowePuttingEndEndtoEnd2020}. Similar approaches utilising mutual information maximization for representation learning have also been explored.
	
	Deep InfoMax incorporates an ANN encoder which maximises the mutual information between input and output. This is achieved by incorporating knowledge about locality in the representations, resulting in locally-consistent information across structural locations \cite{hjelmLearningDeepRepresentations2019}. Meanwhile, InfoGAN, C-DSVAE and S3VAE each introduce a different flavour on the mutual information maximisation scheme, while obtaining disentangled representations \cite{chenInfoGANInterpretableRepresentation2016, baiContrastivelyDisentangledSequential2021, zhuS3VAESelfSupervisedSequential2020}. In addition, InfoGAN's learning approach results in interpretable representations. Manipulating an individual dimension in the latent space results in changes to a specific feature of the generated data while leaving other features unaffected. Building upon these concepts, Bridge-GAN introduces an intermediate latent space, or ``bridge" between text and images. This approach enables the synthesis of interpretable and disentangled representations for text-to-image synthesis \cite{yuanBridgeGANInterpretableRepresentation2020}.
	
	Continuing in the line of interpretable representation learning, Timeline uses recurrent neural networks with an attention mechanism to aggregate sequential health data to interpretable representations \cite{baiInterpretableRepresentationLearning2018}. Its interpretability is achieved through analysis of the weights associated with different medical codes. Similarly, Agrawal and Ganapathy apply relevance weighting on raw speech data, allowing for interpretation of the representations during forward propagation \cite{agrawalInterpretableRepresentationLearning2020}. 
	
	Lastly, InfoVAEGAN combines the framework of Generative Adversarial Networks with concepts from VAEs, enabling the learning of interpretable data variations \cite{yeInfoVAEGANLearningJoint2021}.
	
	

\section{Regularisation}
	The practice of adding a penalty term to the loss function, known as regularization, has been widely used for various purposes. Kukačka and Golkov provide a survey categorising different regularisation techniques, including those that impose constraints on the weights, or on the activations \cite{kukackaRegularizationDeepLearning2017}.
	
	Regularisation terms that enforce constraints on the weights typically aim to improve generalisation performance by penalising complexity. Weight decay, for example, achieves this by applying the $l^2$-norm on the network weights, encouraging smaller weights \cite{gneccoWeightdecayTechniqueLearning2009}. Another approach described by Kukačka and Golkov is weight smoothing which applies $l^2$-norm to the gradients during training. Weight elimination is similar to weight decay but favours sparse networks \cite{weigendGeneralizationWeightEliminationApplication1990}. Soft weight-sharing clusters weights together, ensuring that weights within a cluster have similar values \cite{nowlanSimplifyingNeuralNetworks1992}.
	
	Tian and Zhang discuss sparse vector-based regularisation, which imposes constraints on the activations. This is useful for applications requiring sparse representations, such as data compression \cite{tianComprehensiveSurveyRegularization2022}. Continuing in the line of activation regularisation, Tomczak introduces a regularisation term that encourages activations to maximise entropy, resulting in uncorrelated and disentangled representations \cite{tomczakLearningInformativeFeatures2016}. Meanwhile, Wu et al. introduce a regularisation term putting constraints on the output activations to improve the interpretability of representations \cite{wuImprovingInterpretabilityRegularization2018}.


\section{Alternative priors and posteriors in VAEs}
	Taking inspiration from VAEs, V-GIM minimises the KL-divergence with its posterior distribution and a fixed prior $\prior=\standardnormal$. This results in a latent space that can be better understood. In recent years, several contributions have been made to VAEs concerning different priors or posteriors.
	
	In VAEs, the posterior $\qphizx$ serves as an approximation to the true posterior $\pphizx$ \cite{odaiboTutorialDerivingStandard2019}. This approximate posterior is most commonly chosen as a simple factorised Gaussian for mathematical convenience. However, this is often an oversimplification of the true posterior \cite{nalisnickApproximateInferenceDeep}. Kingma and Welling demonstrate that the approximate posterior can be extended to a Gaussian with a full covariance matrix \cite{kingmaIntroductionVariationalAutoencoders2019}. Additionally, Nalisnick et al. propose a Gaussian mixture model, which combines several Gaussians, as an approximate posterior, enabling the modelling of multimodal posterior distributions \cite{nalisnickApproximateInferenceDeep}.
	
	Continuing in the exploration of Gaussian mixture models, alternative priors have also been investigated. Guo et al. and Lee et al. experiment with Gaussian mixture model priors \cite{guoVariationalAutoencoderOptimizing2020, leeMetaGMVAEMixtureGaussian2021}, resulting in improved performance. Additionally, Tomczak and Welling introduce VampPrior, choosing the prior as a mixture of variational posteriors. This approach improves performance and mitigates issues related to useless dimensions \cite{tomczakVAEVampPrior2018}, which is a well-known problem in VAEs and also observed in V-GIM.
	









%We have studied the latent representations obtained from maximising the InfoNCE objective. 
%We achieved this through the introduction of V-GIM, a self-supervised representation learning approach with the same InfoNCE objective, but with an additional constraint to the latent space resulting in better interpretable representations. Such that a decoder could be trained and predict meaningful ...



%\section{Explainable AI}
%	%While multiple XAI techniques exist, they work in different paradigms, usually attempt to visualise 
%	This is a vastly different approach from existing techniques in explainable AI. Bai et al. group the techniques in three categories \cite{baiExplainableDeepLearning2021}; attribution-based methods, non-attribution-based methods and uncertainty quantification.
%	1) tries to attribute a prediction to its input features. eg used for images and can highlight regions contribute to the decision.
%	
%	
%	
%	X et al. group XAI techniques in x cateogories
%		In the field of Explainable AI multiple paradigms exist, ranging from activation heatmaps ...
%	These techniques give insights in visual domain, but lack in other domains such as speech domain where heatmaps be harder to gain insights from.


%	
%	- 
%
%---
%Explainable ANNs:
%	- diff paradigms, by looking at heath maps and lr etc
%	- our work is in fact new paradigm, by adding constraints to the optimisation metric, resulting in better understandable latent representations
%	- Explainable deep learning methods survey: \cite{baiExplainableDeepLearning2021} (attribution and non attribution, zie mijn draft.dox) --> probeert contribution van elke feature te linken. dat zijn technieken die werken voor foto's of feature vectors, maar voor puur sequential audio is moeilijker bruikbaar.
%	
%	Maybe exists other techniques that change the ANN resulting in better explainable. (eg pruning?)
%	- methodology to remove features that do not contribute to accuracy. (feature selection) with interpretability motivations. \cite{glorfeldMethodologySimplificationInterpretation1996}
%	
%	
%Explainable learning in speech data
%
%
%Summarise: our method: sequential data/speech data, interpretable, representation learning, disentanglement
