\begin{figure}[h]
	\centering
	\tikzstyle{arrow} = [thick,->,>=stealth]
	\begin{tikzpicture}[
		AnnNode/.style={trapezium, draw=black,
			trapezium stretches=true,
			minimum width=2cm, 
			minimum height=1.5cm,
			rotate=-90,
			trapezium angle=75,
			very thick},
		]
		
		\node[AnnNode] (enc) {\rotatebox{90}{$E(\xith)$}};
		\node[AnnNode] (dec) [right of=enc, xshift=1.5cm] {\rotatebox{90}{$D(\zith)$}};
		
		% enc edges
		\draw[->] ++(-2.5, 0) -- (enc.south) node[above, midway] {$\xith \in \D$};
		\draw[->] 
		[transform canvas={yshift=.7em}] 
		(enc.north) -- ++(1.5, 0) node[above, midway] {$\mui$};
		\draw[->] 
		[transform canvas={yshift=-.7em}] 
		(enc.north) --  ++(1.5, 0) node[below, midway] {$\sigmai$};
		
		\node 
		[transform canvas={xshift=5.5em}] 
		(bracket1) at (enc.north) {\Huge \} };
		
		\node 
		[transform canvas={xshift=10em}] 
		(sample z) at (enc.north) {$\sample{\zith}{\qvaeEmptyZandXI}$ };
		
		
		
		% dec edges
		\draw[->] ++(-2.5, -2.5) -- (dec.south) node[above, midway] {$\zith$};
		\draw[->] 
		(dec.north) -- ++(1.5, 0) node[above, midway] {$\vecti{\tilde{x}}$};	  		
		
		
		
	\end{tikzpicture}
	\caption{High level view of a VAE.}
	\label{fig:vae-repr}
\end{figure}





% from img
% Both blocks depict a neural network. The upper block is the encoder and the lower block the decoder. The upper block receives a data points $\vect{x}$ and produces the parameters of $\qphizx$. Since we choose to model $\qphizx$ as a Gaussian with independent components, the covariance matrix $\covariancemtx$ is zero everywhere except for the diagonal. This way the diagonal values, representing the standard deviations, can be represented via a single vector $\sigmai$. The vectors $\mui$ and $\sigmai$ are $\mu(\vecti{x})$ and $\sigma(\vecti{x})$, respectively. These are the output of the encoder block and form the parameters for $\qphizx$. A single neural network with parameter weights $\phi$ is used to simulate $\qphizx$ for every $\vecti{x} \in \mathcal{D} $. This strategy of sharing $\phi$ across data points is refered to as "amortised variational inference" \cite{kingmaIntroductionVariationalAutoencoders2019}.