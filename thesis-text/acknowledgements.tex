\chapter*{Preface}

I am proud to share with you the final milestone of my academic journey as a Computer Science student, specialised in Artificial Intelligence at Vrije Universiteit Brussel. As you will see, this thesis heavily relies on mathematics. Ironically, in high school, I used to struggle with mathematics and tried to avoid it whenever possible. That is why I pursued a professional bachelor's degree in Applied Information Technology at a college that did not include any mathematics courses.

However, during my bachelor's, I gained a strong interest in machine learning. The idea of a computer learning to perform tasks without explicit instructions fascinated me. But to delve deeper into this field, I realised I needed to catch up on my weak mathematics background. It was a major challenge for me, but slowly and steadily, I began to grasp the concepts and even found myself enjoying it. Despite initially feeling overwhelmed, each "aha" moment brought a sense of satisfaction that made me appreciate mathematics more and more. Today, I can proudly say that I have overcome this challenge, resulting in this thesis on Representation Learning with latent space constraints.

However, I couldn't have come this far without the support of those around me. I am immensely grateful to Prof. Dr. Bart de Boer, my thesis supervisor, for his guidance and engaging discussions. Despite my ever-changing ideas, he steered me towards a focused topic while encouraging my creativity and providing realistic feedback. I would also like to express my gratitude to Prof. Dr. Bart Dhoedt, who taught the Deep Generative Models course at UGent. Your course and the interesting discussions we had, have given me great insights, ultimately improving the quality of my thesis. Lastly, I want to acknowledge the creators of OpenAI. Ironically, this thesis is not only about AI but is also partly written with the assistance of AI. ChatGPT, in particular, served as an advanced grammar checker and contributed to stylistic improvements throughout this work.







%\chapter*{Preface}
%
%I am proud to share with you the final milestone of my academic journey as a Computer Science student, specialised in Artificial Intelligence at Vrije Universiteit Brussel. As you will see, this thesis heavily relies on mathematics. Ironically, in high school, I used to struggle with mathematics and tried to avoid it whenever possible. That is why I pursued a professional bachelor's degree in Applied Information Technology at a college that did not include any mathematics courses.
%
%However, during my bachelor, I gained a strong interest in machine learning. The idea of a computer learning to perform tasks without explicit instructions fascinated me. But to delve deeper into this field, I realised I needed to catch up on my weak mathematics background. It was a major challenge for me, but slowly and steadily, I began to grasp the concepts and even found myself enjoying it. Despite initially feeling overwhelmed, each ``aha" moment brought a sense of satisfaction that made me appreciate mathematics more and more. Today, I can proudly say that I have overcome this challenge, resulting in this thesis on Representation Learning with latent space constraints.
%
%However, I couldn't have come this far without the support of those around me. I am immensely grateful to Prof. Dr. Bart de Boer, my thesis supervisor, for his guidance and engaging discussions. Despite my ever-changing ideas, he steered me towards a focused topic while encouraging my creativity and providing realistic feedback. I would also like to express my gratitude to Prof. Dr. Bart Dhoedt, who taught the Deep Generative Models course at UGent. Your course and the interesting discussions we had have have given me great insights, ultimately improving the quality of my thesis. Lastly, I want to acknowledge the creators of OpenAI. Ironically, this thesis not only explores AI but is also partly written with the assistance of AI. ChatGPT, in particular, served as an advanced grammar checker and contributed to stylistic improvements throughout this work.
%
% 
